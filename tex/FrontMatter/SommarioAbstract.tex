% ------------------------------------------------------------------------ %
% !TEX encoding = UTF-8 Unicode
% !TEX TS-program = pdflatex
% !TEX root = ../Tesi.tex
% !TeX spellcheck = en_US
% ------------------------------------------------------------------------ %
%
% ------------------------------------------------------------------------ %
% 	SOMMARIO + ABSTRACT
% ------------------------------------------------------------------------ %
%
\cleardoublepage
%
\phantomsection
%
\pdfbookmark{Sommario}{Sommario}

% ------------------------------------------------------------------------ %
%
\chapter*{Sommario}

Negli ultimi anni, i dispositivi mobili, come smartphone e tablet, sono diventati sempre più popolari, performanti e meno costosi. Ognuno di noi, di questi tempi, utilizza ogni giorno, questo tipo di dispositivi, per svolgere svariate attività. L'evoluzione degli smartphone, ha certamente influenzato le nostre vite, cambiando il modo in cui interagiamo nel mondo reale, ad esempio introducendo nuovi modi di entrare in contatto con altre persone, usando i social network, e offrendo la possibilità di accedere ad Internet, pressoché in ogni momento.\\
\par In questo contesto di continua evoluzione, i dispositivi mobili, che sono stati progettati per essere macchine di calcolo portatili per uso personale, stanno diventando sempre più simili ai normali personal computer, dal momento che hanno sistemi operativi completi e sufficienti capacità di calcolo. Per questo motivo, essi, stanno progressivamente sostituendo i normali computer nello svolgimento di parecchi compiti, poiché sono, appunto, abbastanza potenti per completare diversi incarichi. Ciò che manca al momento, tuttavia, è un modo concreto, per usare questo tipo di dispositivi contemporaneamente, in modo da collaborare assieme per raggiungere un obbiettivo comune, o eseguire applicazioni distribuite, appositamente progettare per essi.\\
\par Questa tesi ha lo scopo di trovare una soluzione che permetta ai dispositivi mobili di cooperare come se stessero eseguendo un sistema operativo distribuito. In questo lavoro, è mia intenzione, trovare un modo per estendere Android, che è un sistema operativo per questo tipo di dispositivi, trasformandolo in un sistema operativo distribuito, senza però dover influenzare i normali meccanismi, con i quali normalmente funziona. Un altro scopo del mio lavoro è, inoltre, quello di proporre, e realizzare un framework, che ho chiamato \textit{Liquid Android}, che sia in grado di supportare gli sviluppatori, nella creazione di applicazioni distribuite per Android. Per raggiungere questi obbiettivi, questo lavoro comprende l'analisi di tutti i requisiti e lo sviluppo di alcuni prototipi.\\
\par A differenza di altre soluzioni, il mio framework, non necessita dei privilegi di \textit{root}, ed è completamente estensibile e open source. Per funzionare, infatti, necessita solamente che i dispositivi coinvolti siano connessi alla stessa rete WiFi.
%
\cleardoublepage
%\vfill
%
% ------------------------------------------------------------------------ %
%
\selectlanguage{english}
%
\pdfbookmark{Abstract}{Abstract}
%
\chapter*{Abstract}
\par In the last few years, mobile devices, such as smartphones and tablets, are becoming increasingly popular, powerful and cheap.
By now, all of us, every day, use these types of devices to carry out all kind of different activities. The smartphone evolution, indeed, has affected our lives, changing real world interaction such as innovative ways to get in touch with other people using social networks, and providing Internet network connectivity almost all the time.\\
\par In this context of continuous evolution, mobile devices which have been designed to be portable computing machines for personal use, are becoming more and more complete personal computers, with fully functional operating systems and large computational capabilities. They are progressively, substituting standard notebooks and desktop in performing many tasks, thus they are powerful enough to perform multi purposes duties. Nevertheless, what is lacking at the moment is a concrete way to use these devices together, in cooperation to achieve a common goal, or run native distributed applications.\\
\par This dissertation aims to find a way to let mobile devices cooperate as if they were running a distributed OS. I want to find a way to extend a mobile operating system, Android, to make it similar to distributed OSs, though without affecting its main working mechanisms. Furthermore, I want to propose a framework, called \textit{Liquid Android}, supporting Android developers to implement native Android distributed applications.  My work analyzes all these requirements and develops some prototypes to reach these objectives.\\
\par A opposed to existing solutions, my framework does not need to force the system to have root privileges, and it is completely extensible and open source. It only needs devices connected to the same standard WiFi network to work.


\medskip
% ------------------------------------------------------------------------ %
%
\selectlanguage{english}
% ------------------------------------------------------------------------ % 