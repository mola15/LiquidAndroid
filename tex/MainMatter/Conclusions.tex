% ------------------------------------------------------------------------ %
% !TEX encoding = UTF-8 Unicode
% !TEX TS-program = pdflatex
% !TEX root = ../Tesi.tex
% !TeX spellcheck = en_US
% ------------------------------------------------------------------------ %
%
% ------------------------------------------------------------------------ %
% 	CONCLUSIONS AND FUTURE WORKS
% ------------------------------------------------------------------------ %
%
\chapter{Conclusions and Future Work}
%
\label{cap:conclusions}
%
% ------------------------------------------------------------------------ %
%
This chapter aims to analyze what has been done in order to give feedback
and understand if the initial objectives have been reached.\\
The chapter is divided into two parts, the first is the conclusion of the work, with
the analysis of what I have reached with my idea and my implementation; the
second part is focused on future implementation.
\section{Conclusions}
Liquid Android middleware aims to enrich the Android OS functionalities by transforming it in to a fully working distributed operating system. My work aims to differentiate itself from existing commercial solutions, seen at the end of the Chapter \ref{cap:statoarte}, in Section \ref{existings} offering a both low cost and easy to deploy solution using a distributed approaches where mobile devices are actors and computing nodes.\\
Liquid Android middleware offers:
\begin{itemize}
	\item automatic network construction and maintenance, when devices are connected in LAN,
	\item same intent resolution mechanism of the standard Android OS,
	\item intent conversion language and automatic tools to perform the conversions,
	\item same standard Android security mechanisms,
	\item development API, easy to use and to be extended to create other purposes distributed systems.
\end{itemize}
Theoretically speaking the goal was reached: having created the networking structure and having defined a new language suitable for the purpose to distribute intents over the network. In this way the Android OS become a distributed OS over a LAN, because any task which can be done by creating implicit intent can be encapsulated in a message and sent, to be executed, on any device with the service active in the network.The research part
was the study and the definition of mechanisms able to let communicate different devices, using a standard mobile operating system. I have defined the steps needed to perform this kind of extension with any existing mobile operating system. My steps, in \ref{cap:proposedsolution} can be used indeed to create similar solutions for other kind of devices. The creation of the middleware and of reliable connections was more part of the implementation, where I did not introduce innovations in the used technologies. That is because reliable and powerful technologies assure the right behavior of the whole architecture.\\
Tests performed in this work aimed to stress the whole system simulating a common scenario with a few number of devices involved: they shown good overall performance of the framework and its well behavior in terms of scalability and dynamicity. Real Android devices behaved as expected under test conditions and the test app built on purpose was always fully usable.\\
I think the work which has been done is positive, interesting and promising: the
idea of having distributed mobile operating system to let users ,use their devices in an environment as they were a single big device, by sending task to any of them connected with my service, is certainly a good feature nowadays in which this kind of devices are becoming more numerous and less expensive. Moreover it is possible to exploit, such a system, to take advantage of the increasing computing capability of mobile devices, and let them cooperate respecting standard OS mechanisms.
\section{Future Work} 
In this section I would like to briefly report and talk about the future developments which can be done on my work, in order to extend and make the framework as complete possible.
My thesis aims to find a unequivocal and uniform way to extend mobile operating system to give them distributed system functionalities. In particular I want to let any kind of device in a network cooperate with each other to perform task. Obviously It was not possible for
reason of time, and complexity to develop a system to be put on the market nowadays, compatible with all the devices. Thus I selected Android devices because they are the most common, cheap, and they use an open source OS. My work can be part of a more complex and complete architecture which can be widely adopted. Following the the very same steps I have done, it would be possible to define a standard language definition to let any mobile devices communicate and cooperate under the same LAN. My JSON solution can be extended and easily used to adapt also IOS and Windows Phone similar intent mechanism, to reach the global goal.
% todo da sistemare



%
% -----------------------------END------------------------------------- %