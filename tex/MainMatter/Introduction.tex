% ------------------------------------------------------------------------ %
% !TEX encoding = UTF-8 Unicode
% !TEX TS-program = pdflatex
% !TEX root = ../Tesi.tex
% !TeX spellcheck = en_US
% ------------------------------------------------------------------------ %
%
% ------------------------------------------------------------------------ %
% 	INTRODUZIONE
% ------------------------------------------------------------------------ %
%
\cleardoublepage
%
\phantomsection
%
\chapter{Introduction}
%
\markboth{Introduction}{Introduction}	% headings
%
\label{cap:introduction}
%
% ------------------------------------------------------------------------ %
%

\section{Motivation}\label{motivation}
\par Nowadays mobile devices have changed the way we approach technologies, they are powerful enough to do most of the things we need in a fast and efficient way, without the need to use a \textit{regular computer} with a \textit{standard desktop operating system}. Mobile operating systems \textit{(mobile OSs)}combine features of a personal computer operating system with other features useful for mobile or handheld use; usually including, and most of the following considered essential in modern mobile systems; a touchscreen, cellular, Bluetooth, Wi-Fi Protected Access,Wi-Fi, Global Positioning System (GPS) mobile navigation, camera, video camera, speech recognition, voice recorder, music player, and so on. By the end of 2016, over 430 million smartphones were sold with 81.7 percent running Android, 17.9 percent running iOS, 0.3 percent running Windows Mobile and the other OSs cover 0.1 percent \cite{james2017percent}.\\
Many people have multiple mobile devices for personal use, and for the reasons stated above it would be great if they could use this wide variety of mobile devices together, equipping their operating systems with services to make them \textit{distributed mobile OSs}\\
 I stated that Android is the most common mobile operating system, it is open source and does not need special developer licenses to build applications. So in this dissertation, I will try to provide a solution to the problem of making mobile operating systems acting as distributed OSs, focusing my attention on Android devices. 
 It is now clear which Android is not only a tiny operating system, but a full functional OS to be used for general purposes. One of the most peculiar characteristic of the Android OS is that it can be installed in a variety of devices such as \textit{"handled"}, like smart-phones and tablets, \textit{"wearable"}, like smart-watches, but also in other kinds of things like standard desktops and laptops, smart-tv and tv boxes, and so on.\\
The great variety of devices described above can run and benefit all the functions of the Android OS which is acknowledged for its ease of use, and the great abundance of applications, with which users can do almost everything. \\
However, one of the greatest limitations of Android is that the system was designed to run on the top of a virtual machine and each  application which can be executed starts a Linux process which has its own virtual machine (VM), so an application code runs in isolation from other apps. This technique is called \textit{"app sandboxing"} and it is used to guarantee a high level of security, because different applications can not read write, or worse steal, data and sensible information from other applications.That is, each app, by default, has access only to the components that it requires to do its work and no more. This creates a very secure environment in which an app can not access parts of the system for which it is not given permission.\\
Under this limitations the Android OS provides a mechanism to make the various components of the applications and the operating system itself communicate : the so called \textit{"intents"}. An intent is an abstract description of an operation to be performed,it provides a facility for performing late runtime binding between the code in different applications. Its most significant use is in the launching of activities. However, even though the intents can be created and resolved within the same android running devices, there is not a mechanism that can send and resolve intents from one devices to another.\\
In a world where computers are everywhere and can do almost everything and can communicate among themselves in different but efficient ways, the fact that android devices are not able to easily exchange intents is a very major limitation to the android users. As we know, our world is fast moving to a world of \textit{"ubiquitous computing"} where there is no longer a single \textit{"fat calculator"} but a variety of multipurpose and specialized devices. In this world of pervasive computation, Android devices are widespread, cheap and powerful enough to do most of the things that we can imagine and would be wonderful if they could be used together in a smart way.
The aim of this thesis work is to study the android framework to find a solution to this problem, and create a middleware to extend the Android OS, creating a distributed system in which intents can be generated from one device and resolved by others in a network environment, for example in \textit{Local Area Network (LAN)}, or to devices in range using a \textit{WiFi direct} protocol. This can help developers build distributed native Android application to exploit the power of any different device running the OS and let the users use their own devices as if they were one single big device.\\
The proposed solution is a \textit{middleware}, which extends the Android OS, turning it into a distributed operating system, by providing the networking structure, the communication language and data management solution. The result is a standard Android application, which operates as a background working middleware which could open the way of native Android distributed applications. Furthermore, by using the same approach, and by following the same steps I used in the solution chapter of this dissertation, it would be possible to let different mobile OS to communicate in this way, and to easily build cross platform distributed systems.

% ------------------------------------------------------------------------ %
%
%
\section{Outline}
%
\par The thesis is organized as follows:
%
% ------------------------------------------------------------------------ %
%
\begin{description}
%
\item[{\hyperref[cap:statoarte]{The second chapter}}] describes the state of art: it provides a full overview on current technologies, ideas and issues. The chapter starts by presenting the Android operating system with a brief history of versions. Then a detailed presentation of Android's framework component is given to the reader, including security model and connectivity functionalities. The chapter continues describing what is a distributed system, listing its main challenges, properties and its working mechanism such as the communication models, and architectural patterns. The third section adds some technical background explanations, it presents the term \textit{Liquid computing}, and other existing technologies which can be useful to understand the problem, the proposed solution and the development of the system. The last section contains a list of already existing solutions, divided in commercial and academical ones.
%
\item[{\hyperref[cap:probanalysis]{In the third chapter}}] I have defined the faced problem, its constraints and its boundaries. The chapter starts with a contextualization of the given problem, giving a brief recap of the state of the art.Then some restrictions are provided, considering only devices in which the developed system could be installed. The chapter continues with the full description of the problem, the main idea and also a working scheme of the component to be developed. Then the problematic scenarios to be studied are presented, including detailed description of what the middleware to be implemented should work in these situations. There is, finally, a list of constraints that the system must meet to be considered a good solution to the given problem.

\item[{\hyperref[cap:proposedsolution]{In the fourth chapter}}] I have analyzed the solution in detail.It can be considered
the core and the central point where the innovation is described. I split the chapter in three main sections. Firstly, I dedicated a few pages to what we can call "General idea" for the solution. Then the section proposed solution contains all the steps and the analysis I performed to solve the given problem. This is the part which introduces the innovation, and provides the theoretical solution. In the last section there is the description of the so-called \textit{Liquid Android API}, which is an Android library I have produced, and used to implement the solution.

\item[{\hyperref[cap:proofofconcept]{In the fifth chapter}}] I focused the attention on the real case study. The chapter starts with an overview of the system developed, a list of non-functional requirements my application must meet and an explanation of the used technologies. Then the focus is shifted to the implementation. The second section of this chapter is fully dedicated to the presentation of my application, there is a complete analysis of my implemented Liquid Android app with a full working demo and live test cases. The last section presents a second application, which is a concrete case of use of my framework, again with a deep description, screenshot and a complete test case.

\item[{\hyperref[cap:conclusions]{In the sixth chapter}}]it is possible to find final considerations about what has been done. In the first section there is the an analysis of what I have achieved and what can be considered promising. Then there are some considerations about what it would be possible to do with my system in future and how it can be extended to become complete.
%
\end{description}
%
% ------------------------------------------------------------------------ %