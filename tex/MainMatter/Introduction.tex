% ------------------------------------------------------------------------ %
% !TEX encoding = UTF-8 Unicode
% !TEX TS-program = pdflatex
% !TEX root = ../Tesi.tex
% !TeX spellcheck = en_US
% ------------------------------------------------------------------------ %
%
% ------------------------------------------------------------------------ %
% 	INTRODUZIONE
% ------------------------------------------------------------------------ %
%
\cleardoublepage
%
\phantomsection
%
\chapter{Introduction}
%
\markboth{Introduction}{Introduction}	% headings
%
\label{cap:introduction}
%
% ------------------------------------------------------------------------ %
%

\section{Motivation}\label{motivation}
\par Nowadays mobile devices have changed the way we approach to technologies, they are powerful enough to do most of the things we need in a fast and efficient way, without the need of use of a \textit{regular computer} with a \textit{standard desktop operating system}. Mobile operating systems \textit{(mobile OSs)}combine features of a personal computer operating system with other features useful for mobile or handheld use; usually including, and most of the following considered essential in modern mobile systems; a touchscreen, cellular, Bluetooth, Wi-Fi Protected Access,Wi-Fi, Global Positioning System (GPS) mobile navigation, camera, video camera, speech recognition, voice recorder, music player, and so on. By the end of 2016, over 430 million smartphones were sold with 81.7 percent running Android, 17.9 percent running iOS, 0.3 percent running Windows Mobile and the other OSs cover 0.1 percent \cite{james2017percent}.\\
Many people have multiple mobile device for personal use, and for the reasons discussed above would be great if people can use this wide variety of mobile devices together equipping their operating systems with services to make them \textit{distributed mobile OSs}\\
 I stated that Android is the most common mobile operating system, it is open source and do not need special developer licenses to build applications. So in this work I will try to provide a solution to the problem of making mobile operating systems acting as distributed OSs, focusing my attention on Android devices. 
 It is now clear which Android is not only a tiny operating system, but a full functional OS to be used for general purposes. One of the most peculiar characteristic of the Android OS is which it can be installed in a variety of devices such as \textit{"handled"}, like smart-phones and tablets, \textit{"wearable"}, like smart-watches, but also in other kind of things like standard desktops and laptops, smart-tv and tv boxes, and so on.\\
The great variety of devices described above can run and benefit all the functions of the Android OS which is acknowledged for its ease of use, and the great abundance of applications, with which users can do almost everything. \\
However one of the greatest limits of Android is that the system was designed to run on the top of a virtual machine and each  application which can be executed starts a Linux process which has its own virtual machine (VM), so an application's code runs in isolation from other apps. This technique is called \textit{"app sandboxing"} and it is used to guarantee an high level of security, because different applications can not read write, or worse steal, data and sensible information from other applications.That is, each app, by default, has access only to the components that it requires to do its work and no more. This creates a very secure environment in which an app can not access parts of the system for which it is not given permission.\\
Under this limitations the Android OS provides a mechanism to make communicate the various component of the applications and the operating system itself : the so called \textit{"intents"}. An intent is an abstract description of an operation to be performed,it provides a facility for performing late runtime binding between the code in different applications. Its most significant use is in the launching of activities. However, even do the intents can be created and resolved within the same android running devices, there is not a mechanism that can send and resolve intents from a devices to another one.\\
In a world where computers are everywhere and can do almost everything and can communicate among them in different but efficient ways, the fact that android devices are not able to easily exchange intents is such a major limitation to the android users. As we know our world is fast moving to a world of \textit{"ubiquitous computing"} where there is no more a single \textit{"fat calculator"} but a variety of multipurpose and specialized devices. In this world of pervasive computation, Android devices are widespread, cheap and powerful enough to do most of the things that we can imagine and would be great if they can be used together in a smart way.
The aim of this thesis work is to study enough the android framework to find a solution to this problem, and create a middleware to extend the Android OS, creating a distributed system in which intents can be generated from one device and resolved by others in a net connected in a LAN. This can help developers build distributed native Android application to exploit the power of any different device running the OS and let the users use their own devices such as they were one single big device.
\\Each sentence or technology, that may appear not clearly explained here for the reader, is further discussed and clarified in next chapters.

% ------------------------------------------------------------------------ %
%
%
\section{Outline}
%
\par The thesis is organized as follows:
%
% ------------------------------------------------------------------------ %
%
\begin{description}
%
\item[{\hyperref[cap:statoarte]{In the second chapter}}] the state of art is described: a full overview on current technologies, ideas and issues is provided. The chapter starts presenting the Android operating system with a brief history of versions. Then a deep presentation of Android's framework component is give to the reader, including security model and connectivity functionalities. The chapter continue describing what is a distributed system, listing its main challenges, properties and its working mechanism such as the communication models, and architectural patterns. The final section explains the term \textit{Liquid computing}, presenting some existing technologies which can be useful to understand the problem an then the proposed solution and development.
%
\item[{\hyperref[cap:probanalysis]{In the third chapter}}] I have defined the faced problem, its constraints and its boundaries. The chapter starts with a contextualization of the given problem, giving a brief recap of the state of the art.Then are provided some restriction, considering only devices in which the developed system could be installed. The chapter continue with the full description of the problem, the main idea and also a working scheme of the component to be developed. Are then presented problematic scenarios to be studied, including detailed description of what the middleware to be implemented should work in these situations. There is, finally, a list of constraints that the system must meet to be considered a good solution to the given problem.

\item[{\hyperref[cap:proposedsolution]{In the fourth chapter}}] 

\item[{\hyperref[cap:proofofconcept]{In the fifth chapter}}]

\item[{\hyperref[cap:conclusions]{In the sixth chapter}}]
%
\end{description}
%
% ------------------------------------------------------------------------ %