% ------------------------------------------------------------------------ %
% Tesi di laurea magistrale di Molinaroli Marco
%
% basato sul modello proposto da Lorenzo Pantieri
% (http://www.lorenzopantieri.net/LaTeX.html)
% e con la possibilità di attivare le impostazioni
% di impaginazione previste dal Politecnico di Milano
%
% versione 1.0	-- 13 luglio 2016
%	- prima versione completa
% ------------------------------------------------------------------------ %
% Le impostazioni specifiche per il Politecnico di Milano
% sono definite sulla base del seguente documento:
% http://www.ingind.polimi.it/cms/file/1262/Norme_per_la_stesura_di_tesi_di_laurea_specialistica.pdf
% già presente nella cartella AltroMateriale
%
% Alcune delle impostazioni sono commentate e sostituite
% da altre ritenute in qualche modo 'migliori'; possono
% essere ripristinate commentando/decommentando i vari comandi
% ------------------------------------------------------------------------ %
% I seguenti commenti speciali impostano:
% - applemac come codifica di input,
% - PDFLaTeX come motore di composizione;
% - Tesi.tex come documento principale;
% - il controllo ortografico italiano per l'editor.
%
% !TEX encoding = UTF-8 Unicode
% !TEX TS-program = pdflatex
% !TEX root = Tesi.tex
% !TeX spellcheck = en_US
% ------------------------------------------------------------------------ %
%
%
% ------------------------------------------------------------------------ %
% 	PREAMBOLO
% ------------------------------------------------------------------------ %
%
\documentclass[12pt,	% 10-11-12pt (12pt preferibile)
	a4paper,		%
	twoside,		% fronte-retro
	openright,		% nuovi capitoli iniziano nella pagina dispari
	titlepage,% 	% nuova pagina dopo il titolo (necessario per frontespizio)
	]{book}
%
% ------------------------------------------------------------------------ %
%
\usepackage[T1]{fontenc}		% codifica di output
%				% N.B. richiede una distribuzione completa di LaTeX
%
\usepackage{mathpazo}  %Font Palatino
\usepackage[utf8]{inputenc}		% codifica di input; anche [latin1] va bene
%				% N.B. va accordata con le preferenze dell'editor
%
\usepackage[english]{babel}	% scelta lingua, sillabazione...
%				% l'ultima lingua (italiano) sarà la predefinita
%
\usepackage{microtype}		% micro-tipografia
%
% ------------------------------------------------------------------------ %
%
% 	LAYOUT - MARGINI - RILEGATURA
%
% -- AUTOMATICO
\usepackage[binding=5mm]{layaureo} 	% margini ottimizzati per l'A4; rilegatura di 5 mm
%
% -- MANUALE (Impostazioni PoliMi)
%\usepackage{geometry}
%
%\geometry{verbose,	% verbose = displays the parameter results on the terminal
%	top=43mm,	% margine superiore (PoliMi=43mm)
%	bottom=44mm,	% margine inferiore (PoliMi=44mm)
%	inner=41mm,	% margine interno pagina (PoliMi=41mm)
%	outer=32mm,	% margine esterno pagina (PoliMi=32mm)
%	bindingoffset=5mm,	% margine per la rilegatura
%	heightrounded}
%
% ------------------------------------------------------------------------ %
%
\usepackage{emptypage}		% pagine vuote senza testatina e piede di pagina
%
\usepackage{indentfirst}		% rientra il primo paragrafo di ogni sezione
%
\usepackage{booktabs}		% tabelle (\toprule, \midrule, \bottomrule)
%
\usepackage{tabularx}		% tabelle di larghezza prefissata
%
\usepackage{graphicx}		% immagini
%
\usepackage[figuresright]{rotating}	% tabelle a 90 gradi
%
\usepackage{subfig}			% sottofigure, sottotabelle
%
\usepackage{caption}		% didascalie
%
\usepackage{listings}		% codici

\lstdefinelanguage{XML}
{
	basicstyle=\ttfamily\footnotesize,
	morestring=[b]",
	moredelim=[s][\bfseries\color{Maroon}]{<}{\ },
	moredelim=[s][\bfseries\color{Maroon}]{</}{>},
	moredelim=[l][\bfseries\color{Maroon}]{/>},
	moredelim=[l][\bfseries\color{Maroon}]{>},
	morecomment=[s]{<?}{?>},
	morecomment=[s]{<!--}{-->},
	commentstyle=\color{Green},
	stringstyle=\color{blue},
	identifierstyle=\color{red}
}

\lstdefinelanguage{JSON}
{
	basicstyle=\ttfamily\footnotesize,
	morecomment=[l]{:},
	morestring=[b]",
	stringstyle=\color{blue},
	commentstyle=\color{black},
	identifierstyle=\color{red}
}


%
\usepackage[font=small]{quoting}	% citazioni
%
\usepackage{amsmath,amssymb}	% matematica
%
\usepackage{mathtools}		% matematica
%
\usepackage{amsthm}		% matematica
%
\usepackage[output-decimal-marker={,}]{siunitx}	% SI (con separatore decimale=virgola)
%
\usepackage[english]{varioref}		% riferimenti completi, con indicazione della pagina (\vref)
%
\usepackage{mparhack,fixltx2e}	% finezze tipografiche (bug fixes di LaTeX)
%
\usepackage{relsize}			% make text larger or smaller than the surrounding text
% 				% \larger[i] \smaller[i]
%
% ------------------------------------------------------------------------ %
%
% 	BIBLIOGRAFIA
%
% adatta lo stile delle citazioni alla lingua corrente del documento
\usepackage[autostyle,english=american]{csquotes}
%
% pacchetto biblatex
%
% STILI di citazione:
% style=numeric-comp,	<-- ufficialmente richiesto dal PoliMi (numeri tra [ ])
% style=philosophy-modern,	<-- autore-anno (meno anonimo, più immediato e più elegante)
%
\usepackage[style=numeric-comp,	% numeric-comp oppure philosophy-modern,
	hyperref,			% riferimenti cliccabili
	backref,			% link alle pagine in cui il riferimento è citato
	natbib, 			% mantiene compatibilità con eventuali comandi natbib
	backend=biber,		% motore bibliografico (v. ArteLatex di Pantieri)
	% defernumbers=true,	 	% riferimenti ordinati in ordine di comparsa
	sorting=none, %comando per ordinare in ordine di apparizione
	]{biblatex}
	
%
\usepackage{bibentry}
%
\addbibresource{Bibliografia.bib}	% database bibliografico
%
% ------------------------------------------------------------------------ %
%
% Per generare effettivamente la bibliografia nel documento
% questa è la sequenza di composizione:
% 1. si compone il documento con LATEX una prima volta;
% 2. si lancia il programma Biber premendo l’apposito pulsante dell’editor;
% 3. si compone il documento altre 2 volte con LATEX (ma anche 3, NdA)
% Tale sequenza deve essere ripetuta solo se vengono fatte modifiche/aggiunte
% al database bibliografico.
%
% ------------------------------------------------------------------------ %
%
\usepackage[dvipsnames]{xcolor}	% colori - 68 colori predefiniti:
% 				% http://en.wikibooks.org/wiki/LaTeX/Colors
%
\usepackage{eurosym}		% simbolo dell'euro
%
\usepackage{hyperref}		% collegamenti ipertestuali
%
\usepackage{bookmark}		% gestione segnalibri del PDF
%
\usepackage{guit}			% simboli del Guit
%
\usepackage{fancyhdr}		% testatine e piede personalizzati
\setlength{\headheight}{15pt}
%
\usepackage{colortbl}		% per colorare i filetti delle tabelle
%
\usepackage[footnote,		% descrizione acronimo fatta a piè di pagina
	smaller,			% acronimo scritto con dimensione ridotta
	]{acronym}		% acronimi
%
\usepackage{multirow}		% celle tabelle alte più di una riga
%
\usepackage{pdfpages}		% inclusione di files pdf esterni

\usepackage{multicol}       %use multi columns

\usepackage{scrextend}

\usepackage{longtable}  % aggiunto da me 

\usepackage{amsmath}

\usepackage{wrapfig}


%
% ------------------------------------------------------------------------ %
% 	PREAMBOLO - SETUP
% ------------------------------------------------------------------------ %
%
% ------------------------------------------------------------------------ %
% !TEX encoding = UTF-8 Unicode
% !TEX TS-program = pdflatex
% !TEX root = Tesi.tex
% !TEX spellcheck = it-IT
% ------------------------------------------------------------------------ %
%
% ------------------------------------------------------------------------ %
% 	PREAMBOLO - SETUP
% ------------------------------------------------------------------------ %
% Comandi personali
% ------------------------------------------------------------------------ %
%
\newcommand{\myName}{Marco Molinaroli}			% autore
\newcommand{\myMatricola}{837721}			% matricola
\newcommand{\myTitle}{Modello di Tesi di Laurea in \LaTeX{}}	% titolo
\newcommand{\myUni}{Politecnico di Milano}		% università
\newcommand{\myFaculty}{Scuola di Ingegneria Industriale e dell'Informazione}	% facoltà/scuola
\newcommand{\myDegree}{Ingegneria Informatica}		% laurea
\newcommand{\myThesis}{Tesi di Laurea Magistrale}	% tipo di tesi
\newcommand{\myDepartment}{Dipartimento di Informazione e Bioingegneria}	% dipartimento
\newcommand{\myProf}{Prof. Luciano Baresi}	% relatore
\newcommand{\myOtherProf}{}		% eventuale correlatore
\newcommand{\myLocation}{Milano}				% dove
\newcommand{\myTime}{April 2017}			% quando
\newcommand{\myAcademicYear}{2015--2016}		% anno accademico
\newcommand{\myLogo}{logoPoliMINuovo.png}	% logo
\newcommand{\myLogoCFD}{logoPoliMiCFD}		% logo CFD :-)
\newcommand{\myUrlUni}{www.polimi.it}			% sito PoliMi
\newcommand{\myUrlFaculty}{www.ingindinf.polimi.it}	% sito Facoltà
%
% ------------------------------------------------------------------------ %
% Impostazioni di amsmath, amssymb, amsthm
% ------------------------------------------------------------------------ %
%
% un ambiente per i sistemi
\newenvironment{sistema}%
	{\left\lbrace\begin{array}{@{}l@{}}}%
	{\end{array}\right.}
%
% epsilon theta rho phi
\renewcommand{\epsilon}{\varepsilon}
\renewcommand{\theta}{\vartheta}
%\renewcommand{\rho}{\varrho}
\renewcommand{\phi}{\varphi}
%
\renewcommand{\vec}{\mathbf} 	% vettori in tondo nero
%
% ------------------------------------------------------------------------ %
% Impostazioni di biblatex
% 
\defbibheading{bibliography}{%
	\cleardoublepage%
	\phantomsection%
	\addcontentsline{toc}{chapter}{\bibname}%
	\chapter*{\bibname\markboth{\bibname}{\bibname}}%
	}
%
% ------------------------------------------------------------------------ %
% Impostazioni di xcolor
% ------------------------------------------------------------------------ %
%
% webcolors
\definecolor{webgreen}{rgb}{0,.5,0}
\definecolor{webbrown}{rgb}{.6,0,0}
%
% BluePolimi (colori delle presentazioni PPT del Politecnico di Milano)
\definecolor{darkbluePoliMi}{rgb}{0,0.18,0.40}	%rgb(0, 46, 103)
\definecolor{midbluePoliMi}{rgb}{0.33,0.47,0.62}	%rgb(84, 121, 157)
\definecolor{lightbluePoliMi}{rgb}{0.53,0.64,0.73}	%rgb(134, 163, 186)
\definecolor{orangePoliMi}{rgb}{1,0.59,0}		%rgb(255, 151, 0)
%
% redSapienza (rosso Sapienza)
\definecolor{redSapienza}{rgb}{0.514,0.031,0.165}	%rgb(131, 8, 42)
%
% ------------------------------------------------------------------------ %
% Impostazioni di listings
% ------------------------------------------------------------------------ %
%
\lstset{
	basicstyle=\smaller[0]\ttfamily,		% Black & White:
	keywordstyle=\color{RoyalBlue},	% keywordstyle=\color{black}\bfseries,
	commentstyle=\color{webgreen},	% commentstyle=\color{gray},
	stringstyle=\color{webbrown},		% stringstyle=\color{black},
	numbers=left,
	numberstyle=\smaller[2],
	stepnumber=1,
	numbersep=8pt,
	showspaces=false,
	showstringspaces=false,
	showtabs=false,
	breaklines=true,
	frameround=ffff,
	frame=single,
	tabsize=2,
	captionpos=t,
	breakatwhitespace=false,
	}
%
% Solution to the encoding issue
\lstset{literate=
  {á}{{\'a}}1 {é}{{\'e}}1 {í}{{\'i}}1 {ó}{{\'o}}1 {ú}{{\'u}}1
  {Á}{{\'A}}1 {É}{{\'E}}1 {Í}{{\'I}}1 {Ó}{{\'O}}1 {Ú}{{\'U}}1
  {à}{{\`a}}1 {è}{{\`e}}1 {ì}{{\`i}}1 {ò}{{\`o}}1 {ù}{{\`u}}1
  {À}{{\`A}}1 {È}{{\'E}}1 {Ì}{{\`I}}1 {Ò}{{\`O}}1 {Ù}{{\`U}}1
  {ä}{{\"a}}1 {ë}{{\"e}}1 {ï}{{\"i}}1 {ö}{{\"o}}1 {ü}{{\"u}}1
  {Ä}{{\"A}}1 {Ë}{{\"E}}1 {Ï}{{\"I}}1 {Ö}{{\"O}}1 {Ü}{{\"U}}1
  {â}{{\^a}}1 {ê}{{\^e}}1 {î}{{\^i}}1 {ô}{{\^o}}1 {û}{{\^u}}1
  {Â}{{\^A}}1 {Ê}{{\^E}}1 {Î}{{\^I}}1 {Ô}{{\^O}}1 {Û}{{\^U}}1
  {œ}{{\oe}}1 {Œ}{{\OE}}1 {æ}{{\ae}}1 {Æ}{{\AE}}1 {ß}{{\ss}}1
  {ç}{{\c c}}1 {Ç}{{\c C}}1 {ø}{{\o}}1 {å}{{\r a}}1 {Å}{{\r A}}1
  {€}{{\EUR}}1 {£}{{\pounds}}1
}
%
% Definizione ambienti per i vari linguaggi
%
\lstnewenvironment{Matlab}{\lstset{language=Matlab}}{}
%
\lstnewenvironment{C++}{\lstset{language=C++}}{}
%
\lstnewenvironment{bash}{\lstset{language=bash}}{}
%
%
% Comando per dare nome alla lista dei codici
%
\addto\captionsitalian{\renewcommand{\lstlistingname}{Codice}}
%
\addto\captionsitalian{\renewcommand{\lstlistlistingname}{Elenco dei codici}}
%
%\renewcommand{\lstlistingname}{Elenco dei codici}
%\renewcommand{\lstlistlistingname}{\lstlistingname}
%
% ------------------------------------------------------------------------ %
% Impostazioni di hyperref
% ------------------------------------------------------------------------ %
%
% per la descrizione delle varie opzioni vedere
% la guida del pacchetto hyperref
%
\hypersetup{
	%hyperfootnotes=false,
	%plainpages=false,
	%pdfpagelabels,
	colorlinks=true,
	linktocpage=true,	% true=link nei numeri pagina / false=link nel titolo
	pdfstartpage=1,
	pdfstartview=FitV,
	breaklinks=true,
	pageanchor=true,
	pdfpagemode=UseOutlines,
	%bookmarksnumbered,
	%bookmarksopen=true,
	bookmarksopenlevel=1,
	hypertexnames=true,
	pdfhighlight=/O,
	urlcolor=webbrown,		% colore dei link a pagine web
	linkcolor=RoyalBlue,		% colore dei collegamenti nel testo
	citecolor=webgreen,		% colore delle citazioni
	pdftitle={\myTitle},		% da qui in poi compilazione metadati
	pdfauthor={\textcopyright\ \myName, \myUni},
	pdfsubject={},
	pdfcreator={pdfLaTeX},
	pdfproducer={LaTeX with hyperref},
	pdfkeywords={polimi,
		tesi,
		latex,
		laurea,
		dottorato,
		scribd},
}
%
% comando per inviare mail
\newcommand{\mail}[1]{\href{mailto:#1}{\texttt{#1}}}
%
% Si possono avere tutti i collegamenti in nero e senza riquadri
% scrivendo semplicemente:
% \hypersetup{hidelinks}
%
% ------------------------------------------------------------------------ %
% Impostazioni di graphicx
% ------------------------------------------------------------------------ %
%
% Elenco dei percorsi in cui saranno cercate le immagini da inserire
%
% In questo modo non è necessario specificare il percorso relativo
% dell'immagine all'interno di \includegraphics{}, ma solo il nome.
%
% N.B. assicurarsi che non siano presenti più immagini
% con lo stesso nome.
%
\graphicspath{
	{Immagini/}
	{Immagini/StateOfArt/}
	{Immagini/Problem/}
	{Immagini/Solution/}
	}
%
% ------------------------------------------------------------------------ %
% Impostazioni di caption
% ------------------------------------------------------------------------ %
%
\captionsetup{tableposition=top,
	figureposition=bottom,
	font=small,
	format=hang,
	labelfont=bf}
%
% ------------------------------------------------------------------------ %
% Impostazioni di fancyhdr
% ------------------------------------------------------------------------ %
%
% Impostazioni preferibili, ma NON del tutto adeguate alle norme POLIMI
% N.B. si possono usare queste impostazioni senza problemi anche per il PoliMi.
%
\pagestyle{fancy}			% sostituisce \pagestyle{header} standard
%
%\renewcommand{\chaptermark}[1]{	% ridefinisce indicazione capitolo
%	\markboth{\chaptername\ \thechapter.\ #1}{}}
%
\makeatletter 			% necessary for using \@chapapp
\renewcommand{\chaptermark}[1]{	% ridefinisce indicazione capitolo
  \markboth{\@chapapp\ \thechapter.\ #1}{}} % distinzione 'Capitolo' / 'Appendice'
\makeatother
%
\renewcommand{\sectionmark}[1]{	% ridefinisce indicazione sezione
	\markright{\thesection.\ #1}}
%
\fancyhf{}				% svuota testatine e piede
%
\fancyhead[LE,RO]{\bfseries\thepage}	% numero pagine in alto
%
\fancyhead[LO]{\bfseries\rightmark}	% info sezione nelle pag. dispari
%
\fancyhead[RE]{\bfseries\leftmark}	% info capitolo nelle pag.pari
%
\renewcommand{\headrulewidth}{0.4pt}	% spessore linea header
%
\renewcommand{\footrulewidth}{0pt}	% spessore linea footer (0pt=nascosta)
%
\fancypagestyle{plain}{				% ridefinizione stile inizio capitolo
		\fancyhead{}			% header vuoto
		\fancyfoot[C]{\bfseries\thepage}		% numeri in grassetto al centro
		\renewcommand{\headrulewidth}{0pt}	% no linea
		}

% Impostazioni degli acronimi
% ------------------------------------------------------------------------ %
%
% descrizione acronimi GIUSTIFICATA
\makeatletter
\def\bflabel#1{{\textbf{\textsf{#1}}\hfill}}
\renewenvironment{AC@deflist}[1]%
{\ifAC@nolist%
\else%
\begin{list}{}%
{\settowidth{\labelwidth}{\textbf{\textsf{#1}}}%
\setlength{\leftmargin}{\labelwidth}%
\addtolength{\leftmargin}{\labelsep}%
\renewcommand{\makelabel}{\bflabel}}%
\fi}%
{\ifAC@nolist%
\else%
\end{list}%
\fi}%
\makeatother
%
% ------------------------------------------------------------------------ %
% Altro
% ------------------------------------------------------------------------ %
%
% Gradiente
\newcommand{\gradiente}[1]{$\nabla #1$}
%
% puntini di omissione [...]
\newcommand{\omissis}{[\dots\negthinspace]}
%
% Eccezioni all'algoritmo di sillabazione
\hyphenation{OpenFOAM}
\hyphenation{Matlab}
\hyphenation{bash}
%
% ------------------------------------------------------------------------ %
% Finezze tipografiche per il Politecnico di Milano
% ------------------------------------------------------------------------ %
%
% Le seguenti modifiche possono essere commentate
% o adeguate ad un'altra università (es. 'Yale Blue'
% per l'università di Yale, 'Rosso Sapienza' per La Sapienza..)
%
% Filetti tabelle colorati
\arrayrulecolor{darkbluePoliMi}
%
%
% Righe delle note a piè di pagina colorate
\renewcommand{\footnoterule}{%
  \kern -3pt
  {\color{darkbluePoliMi} \hrule width 0.4\textwidth}
  \kern 2.6pt
}
%
% ------------------------------------------------------------------------ %		% file con le impostazioni personali
%
%
% ------------------------------------------------------------------------ %
% 	BEGIN DOCUMENT
% ------------------------------------------------------------------------ %
%
\begin{document}
%
% ------------------------------------------------------------------------ %
% 	FRONTMATTER
% ------------------------------------------------------------------------ %
%
\frontmatter
%
%
% --------- CANCELLARE o COMMENTARE ---------------- %
%\input{FrontMatter/Frontespizio}		% frontespizio figo ma non ufficiale al PoliMi
%\input{AltroMateriale/DescrizioneOpera}	% non c'entra nulla con la tesi vera e propria
\cleardoublepage
% ------------------------------------------------------------------------ %
%
% Frontespizio ufficiale del Politecnico di Milano
% ------------------------------------------------------------------------ %
% !TEX encoding = UTF-8 Unicode
% !TEX TS-program = pdflatex
% !TEX root = ../Tesi.tex
% !TEX spellcheck = it-IT
% ------------------------------------------------------------------------ %
%
% ------------------------------------------------------------------------ %
% 	DEDICA
% ------------------------------------------------------------------------ %
%
\cleardoublepage
%
\phantomsection
%
\thispagestyle{empty}
%
\pdfbookmark{Copertina}{Copertina}
%
\vspace*{\stretch{1}}
%

%%%%%%% CONTENUTO %%%%%%%%%%%%
\begin {center}
{
	\fontsize{1cm}{1.2em}\textbf{\textsc{\myUni}}\\[1cm]
}
\end{center}
\begin{center} 
{
	\fontsize{0.6cm}{1em}\textnormal{\myFaculty}\\[0.2cm]
	\fontsize{0.6cm}{1em}\textnormal{Corso di Laurea Magistrale in Ingegneria Informatica}
}
\end{center}

\begin{figure}[h]
	\centering
 	\includegraphics[width=0.7\linewidth]{\myLogo}
\end{figure} 
 
\begin {center}
{
	\fontsize{1.4cm}{1em}\textbf{TITOLO DELLA TESI} \\[3cm]
}
\end{center}

% first column
\begin{multicols}{2}

	\begin{minipage}[t]{0.5\textwidth}
		
		\fontsize{0.6cm}{0.72em}\begin{addmargin}[0em]{1em}\textnormal{Relatore:\\} \end{addmargin}
		
		\fontsize{0.6cm}{0.72em}\begin{addmargin}[0em]{1em}\textbf{\myProf \\[0.5cm]} \end{addmargin}
			

			
	\end{minipage}

\columnbreak

%second column
	\begin{minipage}[t]{0.5\textwidth}
	
		\fontsize{0.6cm}{0.72em}\begin{addmargin}[4em]{1em}\textnormal{Tesi di Laurea di:\\} \end{addmargin}
		
		
		\fontsize{0.6cm}{0.72em}\begin{addmargin}[4em]{1em}\textbf{\myName\\} \end{addmargin}
		
			\fontsize{0.6cm}{0.72em}\begin{addmargin}[4em]{1em}\textnormal{Matricola:\\} \end{addmargin}
		
		
		\fontsize{0.6cm}{0.72em}\begin{addmargin}[4em]{1em}\textbf{837721} \end{addmargin}
		
	\end{minipage}
	
\end{multicols}


\begin {center}
{	
	\text{}\\[2cm]
	\fontsize{0.6cm}{1em}\textnormal{Anno accademico \myAcademicYear}
}
\end{center}
%%%%%%% FINE CONTENUTO %%%%%%%%%%%%

\vspace{\stretch{2}}
%
% ------------------------------------------------------------------------ %
%
% ------------------------------------------------------------------------ %
% !TEX encoding = UTF-8 Unicode
% !TEX TS-program = pdflatex
% !TEX root = ../Tesi.tex
% !TEX spellcheck = it-IT
% ------------------------------------------------------------------------ %
%
% ------------------------------------------------------------------------ %
% 	RINGRAZIAMENTI
% ------------------------------------------------------------------------ %
%
\cleardoublepage
%
\phantomsection
%
\pdfbookmark{Ringraziamenti}{ringraziamenti}
%
\chapter*{Ringraziamenti}
%
\par  Ringraziamenti.\\


\bigskip
\bigskip
\bigskip
\bigskip
\bigskip
\bigskip
\bigskip
\bigskip
\bigskip
\bigskip
 
\noindent\textit{\myLocation, \myTime}
\hfill Marco
%
% ------------------------------------------------------------------------ %
%
% ------------------------------------------------------------------------ %
% !TEX encoding = UTF-8 Unicode
% !TEX TS-program = pdflatex
% !TEX root = ../Tesi.tex
% !TeX spellcheck = en_US
% ------------------------------------------------------------------------ %
%
% ------------------------------------------------------------------------ %
% 	SOMMARIO + ABSTRACT
% ------------------------------------------------------------------------ %
%
\cleardoublepage
%
\phantomsection
%
\pdfbookmark{Sommario}{Sommario}

% ------------------------------------------------------------------------ %
%
\chapter*{Sommario}

Negli ultimi anni, i dispositivi mobili, come smartphone e tablet, sono diventati sempre più popolari, performanti e meno costosi. Ognuno di noi, di questi tempi, utilizza ogni giorno, questo tipo di dispositivi, per svolgere svariate attività. L'evoluzione degli smartphone, ha certamente influenzato le nostre vite, cambiando il modo in cui interagiamo nel mondo reale, ad esempio introducendo nuovi modi di entrare in contatto con altre persone, usando i social network, e offrendo la possibilità di accedere ad Internet, pressoché in ogni momento.\\
\par In questo contesto di continua evoluzione, i dispositivi mobili, che sono stati progettati per essere macchine di calcolo portatili per uso personale, stanno diventando sempre più simili ai normali personal computer, dal momento che hanno sistemi operativi completi e sufficienti capacità di calcolo. Per questo motivo, essi, stanno progressivamente sostituendo i normali computer nello svolgimento di parecchi compiti, poiché sono, appunto, abbastanza potenti per completare diversi incarichi. Ciò che manca al momento, tuttavia, è un modo concreto, per usare questo tipo di dispositivi contemporaneamente, in modo da collaborare assieme per raggiungere un obbiettivo comune, o eseguire applicazioni distribuite, appositamente progettare per essi.\\
\par Questa tesi ha lo scopo di trovare una soluzione che permetta ai dispositivi mobili di cooperare come se stessero eseguendo un sistema operativo distribuito. In questo lavoro, è mia intenzione, trovare un modo per estendere Android, che è un sistema operativo per questo tipo di dispositivi, trasformandolo in un sistema operativo distribuito, senza però dover influenzare i normali meccanismi, con i quali normalmente funziona. Un altro scopo del mio lavoro è, inoltre, quello di proporre, e realizzare un framework, che ho chiamato \textit{Liquid Android}, che sia in grado di supportare gli sviluppatori, nella creazione di applicazioni distribuite per Android. Per raggiungere questi obbiettivi, questo lavoro comprende l'analisi di tutti i requisiti e lo sviluppo di alcuni prototipi.\\
\par A differenza di altre soluzioni, il mio framework, non necessita dei privilegi di \textit{root}, ed è completamente estensibile e open source. Per funzionare, infatti, necessita solamente che i dispositivi coinvolti siano connessi alla stessa rete WiFi.
%
\cleardoublepage
%\vfill
%
% ------------------------------------------------------------------------ %
%
\selectlanguage{english}
%
\pdfbookmark{Abstract}{Abstract}
%
\chapter*{Abstract}
\par In the last few years, mobile devices, such as smartphones and tablets, are becoming increasingly popular, powerful and cheap.
By now, all of us, every day, use these types of devices to carry out all kind of different activities. The smartphone evolution, indeed, has affected our lives, changing real world interaction such as innovative ways to get in touch with other people using social networks, and providing Internet network connectivity almost all the time.\\
\par In this context of continuous evolution, mobile devices which have been designed to be portable computing machines for personal use, are becoming more and more complete personal computers, with fully functional operating systems and large computational capabilities. They are progressively, substituting standard notebooks and desktop in performing many tasks, thus they are powerful enough to perform multi purposes duties. Nevertheless, what is lacking at the moment is a concrete way to use these devices together, in cooperation to achieve a common goal, or run native distributed applications.\\
\par This dissertation aims to find a way to let mobile devices cooperate as if they were running a distributed OS. I want to find a way to extend a mobile operating system, Android, to make it similar to distributed OSs, though without affecting its main working mechanisms. Furthermore, I want to propose a framework, called \textit{Liquid Android}, supporting Android developers to implement native Android distributed applications.  My work analyzes all these requirements and develops some prototypes to reach these objectives.\\
\par A opposed to existing solutions, my framework does not need to force the system to have root privileges, and it is completely extensible and open source. It only needs devices connected to the same standard WiFi network to work.


\medskip
% ------------------------------------------------------------------------ %
%
\selectlanguage{english}
% ------------------------------------------------------------------------ % 
%
% ------------------------------------------------------------------------ %
% !TEX encoding = UTF-8 Unicode
% !TEX TS-program = pdflatex
% !TEX root = ../Tesi.tex
% !TEX spellcheck = it-IT
% ------------------------------------------------------------------------ %
%
% ------------------------------------------------------------------------ %
% 	INDICI
% ------------------------------------------------------------------------ %
%
\cleardoublepage
%
% ------------------------------------------------------------------------ %
%
% Indice Generale
%
\pdfbookmark{\contentsname}{tableofcontents}
%
\setcounter{tocdepth}{2}
%
\tableofcontents
%
\cleardoublepage
%
% ------------------------------------------------------------------------ %
%
% Indice delle Figure
%
\phantomsection
%
\pdfbookmark{\listfigurename}{lof}
%
\listoffigures
%
\cleardoublepage
%
% ------------------------------------------------------------------------ %
%
% Indice delle Tabelle
%
\phantomsection
%
\pdfbookmark{\listtablename}{lot}
%
\listoftables
%
\cleardoublepage
%
% ------------------------------------------------------------------------ %
%
% Indice dei Listati di Programma
%
\phantomsection
%
\pdfbookmark{\lstlistlistingname}{lol}
%
\lstlistoflistings
%
\cleardoublepage
%
% ------------------------------------------------------------------------ %
%
\cleardoublepage
%
% ------------------------------------------------------------------------ %
% 	MAINMATTER
% ------------------------------------------------------------------------ %
%
\mainmatter
%
% ------------------------------------------------------------------------ %
% !TEX encoding = UTF-8 Unicode
% !TEX TS-program = pdflatex
% !TEX root = ../Tesi.tex
% !TeX spellcheck = en_US
% ------------------------------------------------------------------------ %
%
% ------------------------------------------------------------------------ %
% 	INTRODUZIONE
% ------------------------------------------------------------------------ %
%
\cleardoublepage
%
\phantomsection
%
\chapter{Introduction}
%
\markboth{Introduction}{Introduction}	% headings
%
\label{cap:introduction}
%
% ------------------------------------------------------------------------ %
%

\section{Motivation}\label{motivation}
\par Nowadays mobile devices have changed the way we approach technologies, they are powerful enough to do most of the things we need in a fast and efficient way, without the need to use a \textit{regular computer} with a \textit{standard desktop operating system}. Mobile operating systems \textit{(mobile OSs)}combine features of a personal computer operating system with other features useful for mobile or handheld use; usually including, and most of the following considered essential in modern mobile systems; a touchscreen, cellular, Bluetooth, Wi-Fi Protected Access,Wi-Fi, Global Positioning System (GPS) mobile navigation, camera, video camera, speech recognition, voice recorder, music player, and so on. By the end of 2016, over 430 million smartphones were sold with 81.7 percent running Android, 17.9 percent running iOS, 0.3 percent running Windows Mobile and the other OSs cover 0.1 percent \cite{james2017percent}.\\
Many people have multiple mobile devices for personal use, and for the reasons stated above it would be great if they could use this wide variety of mobile devices together, equipping their operating systems with services to make them \textit{distributed mobile OSs}\\
 I stated that Android is the most common mobile operating system, it is open source and does not need special developer licenses to build applications. So in this dissertation, I will try to provide a solution to the problem of making mobile operating systems acting as distributed OSs, focusing my attention on Android devices. 
 It is now clear which Android is not only a tiny operating system, but a full functional OS to be used for general purposes. One of the most peculiar characteristic of the Android OS is that it can be installed in a variety of devices such as \textit{"handled"}, like smart-phones and tablets, \textit{"wearable"}, like smart-watches, but also in other kinds of things like standard desktops and laptops, smart-tv and tv boxes, and so on.\\
The great variety of devices described above can run and benefit all the functions of the Android OS which is acknowledged for its ease of use, and the great abundance of applications, with which users can do almost everything. \\
However, one of the greatest limitations of Android is that the system was designed to run on the top of a virtual machine and each  application which can be executed starts a Linux process which has its own virtual machine (VM), so an application code runs in isolation from other apps. This technique is called \textit{"app sandboxing"} and it is used to guarantee a high level of security, because different applications can not read write, or worse steal, data and sensible information from other applications.That is, each app, by default, has access only to the components that it requires to do its work and no more. This creates a very secure environment in which an app can not access parts of the system for which it is not given permission.\\
Under this limitations the Android OS provides a mechanism to make the various components of the applications and the operating system itself communicate : the so called \textit{"intents"}. An intent is an abstract description of an operation to be performed,it provides a facility for performing late runtime binding between the code in different applications. Its most significant use is in the launching of activities. However, even though the intents can be created and resolved within the same android running devices, there is not a mechanism that can send and resolve intents from one devices to another.\\
In a world where computers are everywhere and can do almost everything and can communicate among themselves in different but efficient ways, the fact that android devices are not able to easily exchange intents is a very major limitation to the android users. As we know, our world is fast moving to a world of \textit{"ubiquitous computing"} where there is no longer a single \textit{"fat calculator"} but a variety of multipurpose and specialized devices. In this world of pervasive computation, Android devices are widespread, cheap and powerful enough to do most of the things that we can imagine and would be wonderful if they could be used together in a smart way.
The aim of this thesis work is to study the android framework to find a solution to this problem, and create a middleware to extend the Android OS, creating a distributed system in which intents can be generated from one device and resolved by others in a network environment, for example in \textit{Local Area Network (LAN)}, or to devices in range using a \textit{WiFi direct} protocol. This can help developers build distributed native Android application to exploit the power of any different device running the OS and let the users use their own devices as if they were one single big device.\\
The proposed solution is a \textit{middleware}, which extends the Android OS, turning it into a distributed operating system, by providing the networking structure, the communication language and data management solution. The result is a standard Android application, which operates as a background working middleware which could open the way of native Android distributed applications. Furthermore, by using the same approach, and by following the same steps I used in the solution chapter of this dissertation, it would be possible to let different mobile OS to communicate in this way, and to easily build cross platform distributed systems.

% ------------------------------------------------------------------------ %
%
%
\section{Outline}
%
\par The thesis is organized as follows:
%
% ------------------------------------------------------------------------ %
%
\begin{description}
%
\item[{\hyperref[cap:statoarte]{The second chapter}}] describes the state of art: it provides a full overview on current technologies, ideas and issues. The chapter starts by presenting the Android operating system with a brief history of versions. Then a detailed presentation of Android's framework component is given to the reader, including security model and connectivity functionalities. The chapter continues describing what is a distributed system, listing its main challenges, properties and its working mechanism such as the communication models, and architectural patterns. The third section adds some technical background explanations, it presents the term \textit{Liquid computing}, and other existing technologies which can be useful to understand the problem, the proposed solution and the development of the system. The last section contains a list of already existing solutions, divided in commercial and academical ones.
%
\item[{\hyperref[cap:probanalysis]{In the third chapter}}] I have defined the faced problem, its constraints and its boundaries. The chapter starts with a contextualization of the given problem, giving a brief recap of the state of the art.Then some restrictions are provided, considering only devices in which the developed system could be installed. The chapter continues with the full description of the problem, the main idea and also a working scheme of the component to be developed. Then the problematic scenarios to be studied are presented, including detailed description of what the middleware to be implemented should work in these situations. There is, finally, a list of constraints that the system must meet to be considered a good solution to the given problem.

\item[{\hyperref[cap:proposedsolution]{In the fourth chapter}}] I have analyzed the solution in detail.It can be considered
the core and the central point where the innovation is described. I split the chapter in three main sections. Firstly, I dedicated a few pages to what we can call "General idea" for the solution. Then the section proposed solution contains all the steps and the analysis I performed to solve the given problem. This is the part which introduces the innovation, and provides the theoretical solution. In the last section there is the description of the so-called \textit{Liquid Android API}, which is an Android library I have produced, and used to implement the solution.

\item[{\hyperref[cap:proofofconcept]{In the fifth chapter}}] I focused the attention on the real case study. The chapter starts with an overview of the system developed, a list of non-functional requirements my application must meet and an explanation of the used technologies. Then the focus is shifted to the implementation. The second section of this chapter is fully dedicated to the presentation of my application, there is a complete analysis of my implemented Liquid Android app with a full working demo and live test cases. The last section presents a second application, which is a concrete case of use of my framework, again with a deep description, screenshot and a complete test case.

\item[{\hyperref[cap:conclusions]{In the sixth chapter}}]it is possible to find final considerations about what has been done. In the first section there is the an analysis of what I have achieved and what can be considered promising. Then there are some considerations about what it would be possible to do with my system in future and how it can be extended to become complete.
%
\end{description}
%
% ------------------------------------------------------------------------ %
%
% ------------------------------------------------------------------------ %
% !TEX encoding = UTF-8 Unicode
% !TEX TS-program = pdflatex
% !TEX root = ../Tesi.tex
% !TEX spellcheck = it-IT
% ------------------------------------------------------------------------ %
%
% ------------------------------------------------------------------------ %
% 	STATO DELL'ARTE
% ------------------------------------------------------------------------ %
%
\chapter{State of the Art}
%
\label{cap:statoarte}
%
% ------------------------------------------------------------------------ %
%
\section{Android OS} \label{androidos}
\par As already mentioned in \ref{motivation}, the Android operating system is an open source OS developed by Google based on Linux kernel, that can be installed on many different kind of devices.\\
In this section i want to give to the reader the basic knowledge of the Android framework to understand why and how the operating system works.
\subsection{Brief History} \label{briefhist}
\par
The Android era officially began on October 22nd, 2008, when the \textit{T-Mobile G1} launched in the United States \cite{verge2011android}.\\
At that moment the company of mountain view, Google, felt the need to create a new operating system which was able to be installed on most modern mobile phones of the time. To meet this need the Google engineers created an OS that was based on the Linux kernel, lightweight enough and ease to be used with simple hand gestures by touching the screen of the phone.\\

\begin{figure}[h]
	\centering
	\includegraphics[width=0.8\textwidth]{T-MobileG1Android1menu}
	\caption{The T-Mobile G1 and the Android 1.0 menù}
	\label{2.1:The T-Mobile G1 and the Android 1.0 menù}
\end{figure}

The main characteristic of the OS were and are also now:
\begin{itemize}
	\item The pull-down notification window.
	\item Home screen widgets.
	\item The Android Market.
	\item Google services integration (eg. Gmail).
	\item Wireless connection technologies (eg Wi-Fi and Bluetooth)
\end{itemize}
The success of the first version of the brand new mobile operating system and the open source philosophy guaranteed the fast spread of the Android devices all over the world. In few years Google improved and released many version of the OS and with the help of the market growth Android has become a complete os. 
In the table below there is a brief description of the various distribution of the Android OS at the time of writing of this document.\\
\begin{table}[h]
	%
	\caption{Android versions}
	%
	\label{tab:vers}
	%
	\centering
	%
	\begin{tabular}{llll}
		%
		\toprule
		%
		\textbf{Name} & \textbf{Version}  & \textbf{Release Date} & \textbf{API Level}\\
		%
		\midrule
		%
		Alpha &	1.0 & September 23, 2008 & 1 \\
		Beta & 1.1 & February 9, 2009 & 2 \\
		Cupcake & 1.5 & April 27, 2009 & 3 \\
		Donut &	1.6 & September 15, 2009 & 4 \\
		Eclair & 2.0 – 2.1 & October 26, 2009 & 5–7 \\
		Froyo & 2.2 – 2.2.3 & May 20, 2010 & 8 \\
		Gingerbread & 2.3 – 2.3.7 & December 6, 2010 & 9–10 \\	
		Honeycomb & 3.0 – 3.2.6 & February 22, 2011 & 11–13 \\
		Ice Cream Sandwich & 4.0 – 4.0.4 & October 18, 2011 & 14–15 \\
		Jelly Bean & 4.1 – 4.3.1 & July 9, 2012 & 16–18 \\
		KitKat & 4.4 – 4.4.4 & October 31, 2013 & 19 \\
		Lollipop & 5.0 – 5.1.1 & November 12, 2014 & 21–22 \\
		Marshmallow & 6.0 – 6.0.1 & October 5, 2015 & 23 \\
		Nougat & 7.0 – 7.1.1 & August 22, 2016 & 24–25 \\
		%
		\bottomrule
		%
	\end{tabular}
	%
\end{table}
 As we can see in \tablename~\ref{tab:vers} there are, currently, 25 level of the Android \textit{API} (Application programming interface
 ) which developers can use to build Android applications. In particular various API levels introduce innovations in the OS but, applications developed using an higher \textit{API level} can not be executed in a device running lower versions of the operating system. This is a second major limitations for the \textit{"Android ecosystem"}, moreover as mentioned before, the Android OS is released under an open source license, which is great for the developer, but which prevents Google to provide updates, in a centralized way, to all devices. For this reason there are currently many active devices running different versions of the mobile OS, as we can check in \tablename~\ref{tab:chart}, which shows, in percentage, the fragmentations of active machines running Android OS.\\
 \begin{table}[h]
 	%
 	\caption{Android OS versions fragmentation}
 	%
 	\label{tab:chart}
 	%
 	
 	%
 	\begin{minipage}{0.5\textwidth}
 		\centering
 		\includegraphics[width=0.9\textwidth]{androidversionchart}
 		 \captionof{figure}{Android OS fragmentation chart}
 		\label{2.2:Android fragmentation chart}
 		
 	\end{minipage}
 ~\hfill~
 \begin{minipage}{0.5 \textwidth}
 	\centering
 	\begin{tabular}{lll}
 		%
 		\toprule
 		%
 		\textbf{Version}  & \textbf{API Level} & \textbf{Distribution}\\
 		%
 		\midrule
 		%
 		2.2 & 8 & 0.1\% \\
 		2.3.3 - 2.3.7 & 10 & 1.2\% \\
 		4.0.3 - 4.0.4 & 15 & 1.2\% \\
 		4.1.x & 16 & 4.5\% \\
 		4.2.x & 17 & 6.4\% \\
 		4.3 & 18 & 1.9\% \\
 		4.4 & 19 & 24.0\% \\
 		5.0 & 21 & 10.8\% \\
 		5.1 & 22 & 23.2\% \\
 		6.0 & 23 & 26.3\% \\
 		7.0 & 24 & 0.4\% \\
 		%
 		\bottomrule
 		%
 	\end{tabular}
\end{minipage}
 	%
 \end{table}
Data in \tablename~\ref{tab:chart} were collected during a 7-day period ending on December 5, 2016, by Google. Any versions with less than 0.1\% distribution are not shown \cite{devandroiddash}.
\subsection{Structure}
\par
Android is an operating system based on the Linux kernel. The project responsible for developing the Android system is called the \textit{Android Open Source Project (AOSP)} and it lead by Google.
\begin{figure}[h]
	\centering
	\includegraphics[width=0.8\textwidth]{oslevels}
	\caption{Android OS 4 layers}
	\label{fig:2.3}
\end{figure}

The OS can be divided into the four layers as depicted the \figurename~\ref{fig:2.3}. An Android application developer typically works with the two layers on top to create new Android applications \cite{vogel2016android}.

\paragraph{Linux Kernel} is the most flexible operating system that has ever been created. It can be tuned for a wide range of different systems, running on everything from a radio-controlled model helicopter, to a cell phone, to the majority of the largest supercomputers in the world \cite{hartman2006linux}. This is in practice the communication layer for the underlying hardware.
\paragraph{Runtime and Libraries} \par Runtime is the term used in computer science to designate the software that provides the services necessary for the execution of a program.There are two different \textit{"runtime systems"} which can work with the Android OS:
\begin{itemize}
	\item \textit{Dalvik VM} is an optimized version for low memory devices of the \textit{Java Virtual Machine (JVM)} used in Android 4.4 and earlier version. It is stack based and it works by converting using a \textit{just-in-time (JIT)}, each time an application is executed, Android's \textit{bytecode} into machine code.
	\item \textit{ART (Android Runtime)} introduced with Android 4.4 KitKat. This runtime uses an \textit{AOT (Ahead-of-Time)} approach, with which code is compiled during the installation of an application and then is ready to be executed.
\end{itemize}
\par Standard Android libraries are for many common framework functions, like, graphic rendering, data storage, web browsing. \cite{vogel2016android}. This layer contains also standard \textit{java libraries}.
\paragraph{Application Framework}\label{appframework} is the layer that contains the Android components for the application such as activities, fragments, services and so on. 
\paragraph{Applications} are pieces of software written in \textit{java code} running on top the other layers.

\subsection{Application Framework}
In this section I want to give some details of the application composition and work flow to better understand the subsequent sections in which I will describe the given problem and the proposed solution.\\
As briefly described in \ref{appframework} the Android application framework \textit{("AppFramework")} is the core of the Android \textit{development API}. It contains useful and needed components to build native apps.\\
The main components with which each application is composed are:

\paragraph{Intents} are objects that initiate actions from other app components, either within the same program \textit{(explicit intents)} or through another piece of software on the device \textit{(implicit intents)}.
Acconrding to the official Google's Android for developer documentation, an Intent is a sort of messaging object which can be used to request an action from another application component (eg. activities). There are three fundamental use cases:
\begin{itemize}
	\item Starting an activity: we will see that activities represents a single screen in Android applications, intents allow to start activities by describing them and carrying any necessary data.
	\item Starting a service: I will explain later in deeper details that services are component which performs operations in background. As for the activities, services are initialized through intent and in the same way they describe the service to start and carries any necessary data.
	\item Delivering a broadcast: broadcast is a message that any app can receive. The system delivers various broadcasts for system events, such as when the system boots up or the device starts charging.
\end{itemize}

As already mentioned there are mainly two categories of intents:
\begin{itemize}
	\item explicit intents, used when it is needed to start component within the same application. As the name implies explicit intents call components by using by name (the full \textit{class object} name), for example, it is possible to start a new activity in response to a user action or start a service to download a file in the background.
	\item implicit intents do not name a specific component, but instead declare a general action to perform, which allows a component from another app to handle it. For example, if you want to show the user a location on a map, you can use an implicit intent to request that another capable app show a specified location on a map \cite{devandroidintent}.
\end{itemize}

\begin{figure}[h]
	\centering
	\includegraphics[width=0.9\textwidth]{intentresolution}
	\caption{Intent resolution mechanism}
	\label{fig:2.4}
\end{figure}

The \figurename~\ref{fig:2.4} explains well how an intent is resolved by the OS whether it is implicit or explicit. When an implicit intent needs to be resolved, the OS searches applications which can handle it by means of \textit{intent filters}.A Intent filter specifies the types of intents that an activity, service, or broadcast receiver can respond to. The Android System searches all apps for an intent filter that matches the intent to be resolved. When a match is found, the system starts the matching component, or, if there are more than one, let the user select the preferred action to be performed.

\paragraph{Activities} are one of the fundamental building blocks of apps on the Android platform. They serve as the entry point for a user's interaction with an app, and are also central to how a user navigates within an app. \cite{devandroidactivity}. An activity is the entry point for interacting with the user. It represents a single screen with a user interface \textit{GUI}: in this way activities are containers for other Android's GUI elements (eg. buttons, textviews,...).

\paragraph{Services} is a general-purpose entry point for keeping an app running in the background for all kinds of reasons. It is a component that runs in the background to perform long-running operations or to perform work for remote processes. A service does not provide a user interface \cite{devandroifundamentals}. 

\paragraph{Broadcast Receivers} are components that enable the system to deliver events to the app outside of a regular user flow, allowing the app to respond to system-wide broadcast announcements. Because broadcast receivers are another well-defined entry into the app, the system can deliver broadcasts even to applications that aren't currently running \cite{devandroifundamentals}.

\subsection{Security}\label{androidsecurity}
\par
As described in \ref{briefhist} Android was born to be a good mobile OS and it is mainly for this reason that the system is designed to protect personal and sensible data form malicious guys.\\
Like the rest of the system, Android's security model also takes advantages of the security features offered by the Linux kernel. Linux is a \textit{multiuser OS} and its kernel can isolate user data from one another: one user can not access another user's file unless explicitly granted permission. Android takes advantages of this user isolation, considering each application a different user provided with a dedicated \textit{UID (User ID)} \cite{elenkov2014android} Android in fact, is designed for smartphones that are personal devices and do not need, usually, a multi physical user support.
The most important security techniques adopted by Android are:

	\paragraph{Application Sandboxing} Android automatically assigns a unique \textit{AppID} (Linux UID) when an application is installed and then executed that specific app in a dedicated process as that UID. This technique isolate all the applications at process level and additionally each app has permissions to read/write a specific and dedicated directory.
	
	\paragraph{Permissions} Since application are sandboxed and do not have the rights to read/write date outside them, it is possible to grant additional rights to android applications by explicitly asking them. Those access rights are called \textit{permission}. Applications can request permissions by listing them in a configuration file called \textit{android manifest}. In Android 5.1 and earlier versions permission are inspected and granted at installation time, when the user is alerted with a dialog box in which are listed permissions the application to be installed needs to work properly and when granted cannot be revoked. Starting from android 6.0 permission are asked the first time that an application need them, and when are granted they can be revoked manually in the OS settings for that specific application.
	
		\begin{figure}[h]
		\centering
		\begin{minipage}[c]{.45\textwidth}
			\centering\setlength{\captionmargin}{0pt}%
			\includegraphics[width=.9\textwidth]{51permissions}
			\caption{Android 5.1- permission example}
		\end{minipage}%
		\hspace{10mm}%
		\begin{minipage}[c]{.45\textwidth}
			\centering\setlength{\captionmargin}{0pt}%
			\includegraphics[width=.9\textwidth]{60permissions}
			\caption{Android 6.0+ permission example}
		\end{minipage}
		\caption{Android permission Examples\label{fig:Andorid permission Examples}}
	\end{figure}

	\paragraph{SeLinux} Security Enhanced Linux, is a \textit{mandatory access control (MAC)} system for the Linux operating system. With a MAC the operating system constrains the ability of a subject or initiator to access or generally perform some sort of operation on an object or target. Starting in Android 4.3, SELinux provides a mandatory access control (MAC) umbrella over traditional discretionary \textit{access control (DAC)} environments. For instance, software must typically run as the root user account to write to raw block devices. In a traditional DAC-based Linux environment, if the root user becomes compromised that user can write to every raw block device. However, SELinux can be used to label these devices so the process assigned the root privilege can write to only those specified in the associated policy.
 	In this way, the process cannot overwrite data and system settings outside of the specific raw block device \cite{secure2017android}.

\subsection{Connectivity}\label{connectivity}
\par
As already amply explained previously many Android design choices are due to the fact that it was thought for mobile devices which must have connectivity to intercommunicate among them.\\
With the evolution of various wireless communication technologies, Android devices, nowadays, are equipped whit different kinds of modulus, the most common are:
\begin{itemize}
	\item Wi-Fi
	\item Bluetooth
	\item NFC
	\item Cellular Network
\end{itemize}
The Android Os provide a full library to operate with these technologies and it is possible to integrate in applications the possibility to communicate over these wireless modules.
With the \textit{Android connectivity API} data can be send and received in an efficient way.\\\\
\par
I have only quickly listed some features and possible issues of my source, to have a complete idea it is possible to read all the official Android documentation in \cite{devandroifundamentals}.
 
\section{Distributed System} \label{distsys}
In this section I want to give to the reader some basics about distributed systems, including technical details and examples to make the proposed solution easier to understand.
\subsection{Definition}\label{distdef}
\textit{A distributed system is a collection of independent computers that appears to its users as a single coherent system.}\\
This definition has several important aspects. The first one is that a distributed system consists of components (i.e., computers) that are autonomous. A second aspect is that users (be they people or programs) think they are dealing with a single
system. This means that one way or the other the autonomous components need to collaborate \cite{tanenbaum2010distributed}.\\
\begin{figure}[h]
	\centering
	\includegraphics[width=0.8\textwidth]{distributedsystem}
	\caption{Distributed system structure}
	\label{fig:2.8}
\end{figure}
In \figurename~\ref{fig:2.8} it is possible to see how can be structured a distributed system: a the top we have the real distributed application, which is the final interface to be used, under which it is possible to have different combinations of services used to make communicate different machines that may use different operating systems. The real magic is done by the layer called \textit{middleware service} in the picture. A middleware in computer science is a set of software which act as intermediaries between structures and computer programs, allowing them to communicate in spite of the diversity of protocols or running OSs.

\subsection{Challenges} \label{chall}
There are many challenges in distributed systems field: distributed applications are often really complex and easily exposed to physical and technical failures because of their nature. Major challenges and property to be considered when developing a system of this kind are:

\begin{figure}[h]
	\centering
	\includegraphics[width=0.8\textwidth]{challengesdistributedsystems}
	\caption{Distributed system challenges}
	\label{fig:2.9}
\end{figure} 

\begin{itemize}
	\item Heterogeneity, is a major challenge because there are many different component to be considered, distributed systems may be developed for example for different hardware, networks, operating systems and programming languages.
	\item Openness, determines whether a system can be extended and reimplemented in various ways, so distributed systems should use standards as much as possible. Developers should always choose the simplest ways during design and implementation phases.
	\item Security, is crucial in many areas of computer science and specially in distributed systems, where data are exchanged by a several number of machines.
	\item Scalability, is the ability to easily increase the size of the system in terms of users/resources and geographic
	span.
	\item Failure handling, is important because having different components working together to a common goal means that distributed system can fail in many ways. This raises some issue: it would be nice id distributed systems can detect, mask and tolerate failures.
	\item Concurrency in distributed systems is a matter of
	fact, access to shared resources (information or services)
	must be carefully synchronized.
	\item Transparency level are listed in \tablename~\ref{tab:transparency}
	\begin{table}[h]
		%
		\caption{Transparency levels}
		%
		\label{tab:transparency}
		%
		\centering
		%
		\begin{tabular}{ll}
			%
			\toprule
			%
			\textbf{Transparency} & \textbf{Description}\\
			%
			\midrule
	%
			Access & Hide differences in data representation and how a resource is accessed\\
			Location & Hide where a resource is located\\
			Migration & Hide that a resource may move to another location\\
			Relocation & Hide that a resource may be moved to another location while in use\\
			Replication & Hide that a resource may be shared by several competitive users\\
			Concurrency & Hide that a resource may be shared by several competitive users\\
			Failure & Hide the failure and recovery of a resource\\
			Persistence & Hide whether a (software) resource is in memory or on disk\\
			%
			\bottomrule
			%
		\end{tabular}
		%
	\end{table}
\end{itemize}

\subsection{Comunication Model}
\paragraph{Remote procedure call (RPC)}
\paragraph{Remote method invocation (RMI)}
\paragraph{Message oriented}

\subsection{Architectures}
\par
There are actually many different kinds of distributed systems which can be classified in by means of their architecture composition.
\paragraph{Client-Server} is the most common architecture in computer systems, there are many variants depending on the internal division of its components but it has a common separation of duties. Server side components are passive and wait for clients invocations.Client computers provide an interface to allow a computer user to request services of the server and to display the results it returns. Servers wait for requests to arrive from clients and then respond to them. Ideally, a server provides a standardized transparent interface to clients so that clients need not be aware of the specifics of the system (i.e., the hardware and software) that is providing the service. The communication adopted by these kind of systems is message oriented or through RPC.
\begin{figure}[h]
	\centering
	\includegraphics[width=0.8\textwidth]{clientserver}
	\caption{Client server architecture}
	\label{fig:2.10}
\end{figure} 
\paragraph{Peer-to-Peer (P2P)} is a fully distributed architecture which in contrast to client-server has not a centralized service provider. Peers are both clients and servers themselves, P2P promotes sharing of resources and services trough direct exchange between peers. Compared to a centralized client-server architecture a P2P net scales better and typically does not have a single point of failure. 
\begin{figure}[h]
	\centering
	\includegraphics[width=0.8\textwidth]{p2p}
	\caption{P2P architecture}
	\label{fig:2.11}
\end{figure} 
\paragraph{REST style} Representational State Transfer (REST) is a style of architecture based on a set of principles that describe how networked resources are defined and addressed.An application or architecture considered RESTful or REST-style is characterized by:
\begin{itemize}
	\item state and functionality are divided into distributed resources,
	\item every resource is uniquely addressable using a uniform and minimal set of commands (typically using HTTP commands of GET, POST, PUT, or DELETE over the Internet),
	\item the protocol is client/server, stateless, layered, and supports caching.
\end{itemize}



\paragraph{Event based} is an architecture in which components collaborate by exchanging information about occurring events. In particular components in the net can \textit{pubblish} notifications about the events they observe or \textins{subscribe} to events they are interested to be notified about. This architecture can be fully distributed with all the same nodes or can have some semi-centralized nodes which are specialized in computing events or routing messages. Communication is, in this case, purely message based asynchronous and multicast. 
\begin{figure}[h]
	\centering
	\includegraphics[width=0.8\textwidth]{publishsubscribe}
	\caption{Publish-subscribe architecture}
	\label{fig:2.12}
\end{figure} 

\subsection{Liquid Computation}
\par 





%
% ------------------------------------------------------------------------ %6
%
% ------------------------------------------------------------------------ %
% !TEX encoding = UTF-8 Unicode
% !TEX TS-program = pdflatex
% !TEX root = ../Tesi.tex
% !TeX spellcheck = en_US
% ------------------------------------------------------------------------ %
%
% ------------------------------------------------------------------------ %
% 	Problem analysis and proposed solution
% ------------------------------------------------------------------------ %
%
\chapter{Problem Analysis}
%
\label{cap:probanalysis}
%
% ------------------------------------------------------------------------ %
%

In this chapter the specific problems of this work will be detailed and analyzed, explaining what are the limits and the constraints the challenge has. The chapter starts with a brief recap, followed by the proper definition of what I faced, while in the last part there is a list of constraints my architecture will have fulfilled in order to have a universal and functional solution.
  

\section{Contextualization} \label{facedProblem}
As already explained introducing this thesis work in \ref{motivation}, I studied in deep the Android operating system to find, and later implement a concrete solution, to the problem I will define and describe in dept in this section. All the work done by me is focused on Android because every mobile operating system is different to each other and has proprietary working mechanism which have to be studied separately. Since there are many more Android devices than any other mobile OS, and Android is an open source software and there is no need to buy development licenses or proprietary hardware or software like using for example Apple systems, I decided to work with it, even if by studying another mobile OS and implementing the same concepts of my solution it is certainly possible to achieve the same result i got by working only with Android. 
In the previous chapter, number \ref{cap:statoarte}, I have defined Android OS working mechanism and components, pointing out the main focus on intent generation and resolution mechanism. I have then defined in deep what a distributed system is and should be, explaining connection mechanism architectures and properties.\\ 
The Android OS is a centralized operating system designed for a single physical user, to be used on personal mobile devices such as smartphones and tablets. The result of this Google's ideas is that in contemporary society there is a wide spread of Android devices, which now have computing capacity comparable to normal desktops and notebooks. Many people have multiple devices which they use separately: typically they use smartphones for calls and  work emails and maybe tablets to easily surf the Internet and play games, but what they can not do is use them together to perform a common task easily. Android, in fact, has not been thought to build a real distributed system, the networking functionalities are designed to exchange messages, and to replace standard personal computers in some task as indeed sending emails.
The result is a non collaborative confused cloud of devices, which are connected to the net, but are not really connected themselves to cooperate. Solutions are often partial or proprietary and closed, even if some useful solutions exist.\\
The idea is let android devices collaborate and cooperate in a \textit{Liquid environment} like the one presented in section \ref{liquid computing}. The fundamental requirement is the implementation of an android service, able to build and maintain a distributed net of android devices over a Wi-Fi LAN (Local Area Network), and then let one, or more devices in that net generate Android intents and distribute them one, or more, of the other devices involved. Thus in this chapter I am considering only Android devices that can be connected in a WiFi LAN.\\
After this brief recap of what has been said about the Android OS and distributed systems in the state of the art chapter, here I am trying to define with more precision the problem I am going to face: which its constraints and its possible goals are.

\section{Considered Devices}
As anticipated above, I am going to take into account only devices that can be somehow connected to a LAN, but as described in \ref{connectivity} Android devices are built to be connected to the Internet and most of them comes with a WiFi chip integrated. Another \textit{"little relaxation"} I want to do is linked to the variety of Android OS versions. I want to take into account only devices updated to at least version 4.4 (API level 19). This is due to the fact that starting from Android KitKat (4.4 version) Google brought some important improvements  to the libraries of the framework and in addition, according to the \tablename~\ref{tab:chart}, with this choice it is possible to cover the 84\% of the active Android devices.\\
Having done these clarifications, now I am defining the problem.

\section{Definition} \label{problemdefinition}
\textit{How can we transform a standard mobile OS into a distributed version of it?} This is the general question I want to give an answer in this work.
As already said the Android OS is a pretty closed system itself, the intent resolution mechanism shows how it is difficult to let communicate various components inside a single device. On the other hand it is equally true that Android devices are real powerful modern computers and would be great if somehow it could be possible to have a device able to detect other devices in a LAN send data and task to perform in a transparent way and then get back, if necessary, result or data. Let me be more concrete, often in a home environment there are several android devices, with a distributed intent resolution mechanism it would be possible for example to take a photo from one device with the camera of another one, to generate an intent to open a file on a group of devices simultaneously, to play a video remotely and so on, only by generating intents and then send them to the distributed net. \textit{How can we let multiple android devices act as a single big distributed system?} This is the question that my thesis is trying to answer. My work is a concrete solution, it is about defining and creating a method to distribute android intents from one device to other in a LAN and then let the OS act as usual to manage and resolve them.\\
So I am trying to let different android devices talk by means of distributing intents using  well known architecture: a master component, let me call it \textit{distributed intent generator} acting as client, and a slave component \textit{distributed intent solver} acting as a server. The two components i will realize will be common Android background services registered on the WiFi LAN. Both of these components will result in android applications so that a single device could be used to control others or to be controlled.\\
In \figurename~\ref{fig:3.1} is presented how such a system should work once the net is up. The distributed system in figure is a simplification of what the middleware for distributing intents will do. The architecture will communicate using standard android networking messages that are build on top standard protocols of the ISO/OSI stack.
 
\begin{figure}[h]
	\centering
	\includegraphics[width=.95\textwidth]{scheme}
	\caption{Distributed intent resolution}
	\label{fig:3.1}
\end{figure}

The important point is having a message with a well defined content: it is what the two parts must write and read, so it has to be clear for machines, must be compliant with all the requests of \textit{M2M (Machine-to-Machine) communication}. This type of communications is a constraint of my work and are explained in the next section \ref{problemconstraints}.Another important point is let the Android OS work as it is designed for, the main aim of this thesis work is to build a middleware to let distribute native Android intents over the network. This is a new approach to this problem in fact, there are yet some android applications which let the user send stream or data to other devices in a LAN but, with specialized and ad hoc built messages within the same application context, using a mechanism really close to explicit intent resolution. My middleware is supposed to address the problem using a more general approach and a mechanism equal to implicit intent resolution. What I'm doing is create a system to spread any kind of implicit intent and let the OS react as usual to perform the required action.
It is not even marginal the choice of the type of network to be used in such a system. Android devices are in fact, usually, mobile devices, and for this reason they can be easily moved from one place to another, so the network must take into account this property dynamically react to continuous changes.\\
Next sections will properly define all the constraints of the given problem and propose a solution that fulfills them all.
\section{Probelm scenarios}
As already anticipated with the definition of the problem,  the aim of this thesis is to give the feeling, to users, that they are working with multiple Android devices as if they were one single distributed operating system. I want to analyze some problematic scenarios an then in the next chapter of the thesis provide, if possible a solution to each specific case.

\subsubsection{Background working middleware}\label{middlewarescenario}

\begin{figure}[h]
	\centering
	\includegraphics[width=.85\textwidth]{scenario1}
	\caption{Liquid Android working as stand alone middleware APK}
	\label{fig:3.2}
\end{figure}
 

In the best case, the result to be achieved would be a single Android APK, to be installed on devices as background bunch of services acting as a middleware. Liquid Android services which providing a communication interface can listen distributed events invocations and react to them automatically. The middleware may intercept local intents, find online devices in the distributed network, let the user select on which of them execute the task and send the intent to the selected remote Android device.\\
The \figurename~\ref{fig:3.2} shows a possible UML component diagram of the Liquid Android middleware, the scheme points out the interactions between the Liquid Android application (APK) and other possible applications installed on the device. The group of services intercepts the local intent, created by a different application in the same device, let the user of the system select on which available device execute the task, build and spread the the distributed intent message on the LAN, and when the message arrives at the other device the middleware transform the received message in a local intent to be resolved as usual by the operating system.\\
I want to provide a simple but concrete example to be more precise, every Android OS version comes with a web browser installed as an APK. When an application needs to open a URL with a browser generate an intent to perform such an action, with a middleware as described above it would be possible to click on the URL on a first device and to open the link in a second device having only installed the Liquid Android APK on both devices.

\subsubsection{Development API}\label{devAPI} A second interesting scenario is one in which the Liquid Android middleware could become a ready to use \textit{application programming interface (API)}. By abstracting the underlying implementation and only exposing objects or actions the developer needs, an API reduces the cognitive load on a programmer. By developing the middleware as an API is it possible to give to Android programmers a library to implement easily and faster, native Android distributed applications. The API implemented could be integrated during the development of such applications like other Android libraries to generate one single APK containing also Liquid Android components. For these purposes it is necessary to produce accurate documentation for developer who could use the Liquid Android API.\\
As already done for the previous case, let me make an example.
\begin{figure}[h]
	\centering
	\includegraphics[width=.9\textwidth]{esempio2}
	\caption{Liquid Android API working example}
	\label{fig:3.4}
\end{figure}
In \figurename~\ref{fig:3.4} there is a scheme showing how the middleware could be used to build two different applications with two different packages (apk) including both the Liquid Android API which let them communicate by sending Android intent generated by \textit{Android Apk 1} and then received by \textit{Android Apk 2} installed respectively on two different devices.

\subsubsection{Data management}
\begin{figure}[h]
	\centering
	\includegraphics[width=.9\textwidth]{scenario3}
	\caption{Liquid Android API data management example}
	\label{fig:3.5}
\end{figure}
Last interesting scenario tu study is the data management problem in using such a system. This is a different type of scenario because it involves both previous scenarios. Having a distributed system always raises the problem of distributed data and data consistency. The middleware to be implemented must consider also the possibility to be used to build distributed Android applications in which data are generated somewhere by one device and then they need to be processed for a result by another one. A simple, but not trivial, example could be the case of a distributed calculator. A device acquires data and sends them to another one to be processed and then ask to that device the results. In the Android environment there is not the concept of distributed file system, so data involved in such an application must be considered and efficiently exchanged between devices.


\section{Constraints} \label{problemconstraints}
In this section I would like to list a set of  constraints for the defined problem, that become requirements that the solution must meet. The section should be divided into two parts, the first for the requirements of the network, the second for the ones of the Android distributed intent generator and solver. The two sections are actually closely related so here I preferred to keep the two parts together, analyzing the entire middleware structure.\\ 
Here is the list:

\begin{itemize}
	\item \textit{M2M communication:} M2M communication is defined as a communication in which the two interlocutors are not humans. It is a communication completely handled by machines and computers \cite{cha2009trust}. It can be considered one of the fundamental enabling technologies of this thesis work, it permits object to communicate without humans being involved. In This type of communication the reader of the content is a computer, in this case are Android devices. The content of the messages must be well formed, the middleware must react properly to the event of receiving a distributed intent. So a clear, defined syntax with a well fixed structure  must be set in order to make everything understandable to a computer.
	
	\item \textit{Transparent:} As already widely discussed a middleware is those which do the \textit{magic}. The proposed solution is intended to be transparent to the Android OS and let it work as usual in resolving implicit intents whether they are distributed or not. Moreover as discussed in the chapter \ref{cap:statoarte}, to be more precise in \tablename~\ref{tab:transparency}, a distributed system must be transparent at many level, in this case the middleware must act as resources manager and efficiently mask resources access and location.
	
	
	\item \textit{Lightweight:} Another constraint to my system is the fact that whatever system I choose to be the solution it must be lightweight. This is needed because my system will work on a WiFi LAN. Messages must be encapsulated, serialized from one device and transferred in another one to be deserialized and analyzed to be executed. Messages must be as easy as they can because they are very frequent in such a system.
	
	\item \textit{Modular:} The implementation of the solution must be modular, this is due to the fact that this middleware is intended to be used as is but also to implement easily other kind of native Android distributed system application. Having a modular structure facilitates the specialization of its component and make all the middleware more readable and easy to use. In this way \text{Liquid Android} can be the substructure of other works.
	
	\item \textit{Extensible:} the implemented solution must meet canonical programming principles Extendability is one of the most important properties to take into account when building a computer system, especially when developing a middleware. Liquid Android modules have to be extensible to be improved or adapted to different purposes.
		
	\item \textit{Secure:} Liquid Android middleware must meet standard Android security design principles as described in \ref{androidsecurity}. The implementation must not exceed the limits imposed by the OS, I do not want to break the Android permission scheme and authorization model by \textit{rooting} the operating system, a process with which is possible to perform action as the administrator in Android environment. Rooting Android devices let application overcome the boundaries of standard applications, by letting them read and write data from all the OS. Moreover the middleware operates on mobile devices which usually contains and can manage many sensible and personal data, communications between these devices must be as secure as possible to limit security threats.
	
	\item \textit{Consistent:} Data and accessed resources involved in the system must meet consistency requirements. When developing distributed systems consistency is one of the main issue. The implemented solution must take into account data produced during the use of the system and make them consistent according to a chosen consistency model.
	
	\item \textit{Scalable:} the system to be implemented has not a fixed number of devices involved in. The chosen network architecture must be able to react according to the changes. Android devices are free to join or leave the network any time, and the system should be able to detect and maintain a dynamic network. Scalability is, in fact, the capability of a system, network, or process to handle a growing amount of work, or its potential to be enlarged in order to accommodate that growth \cite{bondi2000characteristics}. 
	
	\item \textit{Concurrent:} another important aspect of distributed systems is concurrency. Concurrency is the decomposability property of a program, algorithm, or problem into order-independent or partially-ordered components or units \cite{lamport1978time}. The implementation of the services must ensure this property to the system. The middleware has to have the capability to handle different requests at the same time and execute task in more than one device simultaneously.
	
\end{itemize}
The listed requirements, as already told in some of them, are, sometimes, general, in the sense that they have to be respected for the final product: a global and complete structure that starts from the construction of the network architecture arrives to the user's interaction activities on Android devices. This is because the problem I am facing is very big and complex, and it is transversal to the existing technologies, so the whole system must work properly. Keeping in mind what I have just stated, some of these constraints become fundamental requirements that my system must meet. My work has to be clear for developers to be used for further implementations of native Android distributed systems, but even if it can be less clear to an average user it must be usable to those wishing to try distributed intents with their own devices in a home LAN.

\bigskip
\bigskip
\bigskip
\bigskip
\bigskip
\bigskip
\par
In the next chapter I am presenting my idea, the \textit{Liquid Android} middleware, the so called solution to the given problem, explaining what I have done, my considerations about the situation here faced.

%
% -----------------------------END------------------------------------- %
%
% ------------------------------------------------------------------------ %
% !TEX encoding = UTF-8 Unicode
% !TEX TS-program = pdflatex
% !TEX root = ../Tesi.tex
% !TeX spellcheck = en_US
% ------------------------------------------------------------------------ %
%
% ------------------------------------------------------------------------ %
% 	PROPOSED SOLUTION
% ------------------------------------------------------------------------ %
%
\chapter{Proposed Solution}
%
\label{cap:proposedsolution}
%
% ------------------------------------------------------------------------ %
%
\par In this chapter I will report the development of the solution step by step, with full use cases. %The chapter starts with a list of similar solutions already developed, highlighting the differences between them and my thesis work.
 I decided to decompose the chapter with tree mains sections: the first explains the choice of the network, the architecture, the naming service, etc. It lays the foundations for the second part: the definition of the Liquid Android middleware, or better the structure of what will do the magic: intercept, encapsulate, spread, and generate distributed intents in the network so made. The two parts are closely related, therefore their relation was taken into account when I made my choice.\\
The real implementation of the valid solution is left for the next chapter.
%
\section{General Idea}
\par To better explain what I consider solution for my work it is important to
understand the playground to my work. As said in the previous Chapter, \ref{cap:probanalysis}, I
am trying to extend the Android operating system by adding some functionalities to make it similar to a distributed OS, without the need of rooting it or change its standard working mechanism and components, staying in the 7th layer of the ISO/OSI stack , the application layer. Using already developed and operating tools, and respecting all the above listed constraints I am going to make mobile devices in a LAN network communicate and cooperate like they were using a single coherent distributed operating system.
In order to understand what is needed and how it is possible to solve the problem
it is fundamental to understand the type of stack and the network structure
we have to face, and standard Android working principle, in particular the intent resolution mechanism already described in \figurename~\ref{fig:2.4}.\\
Only having clear in mind the problem and its structure it is possible to find the best possible solution. In particular it is possible to decompose the main problem in some sub-problems, which can be understood as general steps in doing similar works of extending a mobile OS to become a distributed OS:
\begin{itemize}
	\item Network architecture, that is the structure and the classification of the nodes involved in the distributed system. As previously said it has to be as reliable  as possible and allow dynamic connection due to the fact that nodes are mobile devices and can be easily moved in and out the network range.
	\item Communication model, that is the way in which involved actors perform the communication. It has to be compliant to M2M, and possibly to H2M communication, and as lightweight as possible to allow fast exchange of messages and data between the network nodes.
	\item Data model, as discussed before in the Chapter \ref{cap:statoarte}, when building distributed systems it is also important to guarantee that data are managed correctly by adopting a consistency policy.
\end{itemize}
It is also necessary to identity the main actor involved in the problem, they are mainly two:
\begin{itemize}
	\item Server application, it is the main actor of this thesis work, it must be an Android application which once installed on a compatible Android device can receive resolve and forward Android intents. It contains the logic and the controllers needed to handle the network structure, find other devices in the network, send and receive messages. It is responsible of resolving all the three sub-problems described above. The server application has also the double function of receiving a message from the network and translate it in a local intent to be resolved by the Android operating system, but also it can act as a client by forwarding a received intent from a third party client application to another server in the net, by encapsulating the intent in a network message.
	\item Clients, can be applications developed in several ways, they are those which are asking to the so called servers to complete task for them. In this case clients could be any kind of third party Android application installed in the device, also running the server component, generating implicit Android intents that need to be resolved by the OS. 
\end{itemize}
Once defined the main actors of the problem I am facing, the next step is to understand how they can interact and communicate. As previously described defining the problematic scenarios in the Chapter \ref{cap:probanalysis}, the Liquid Android middleware in the best case would be a system service which users can control to distribute intents in the local network by using the WiFi chip of the devices.
\begin{figure}[h]
	\centering
	\includegraphics[width=.9\textwidth]{esempio1}
	\caption{Liquid Android working example}
	\label{fig:4.2}
\end{figure}
The \figurename~\ref{fig:4.2} shows exactly how the middleware is supposed to work. In the figure the two devices are both supposed to be connected to the same local network \textit{(they are under the same WiFi access point)} and they are executing any standard Android OS version starting from 4.4 KitKat \textit{(API level 19)}. The so called, in the picture, \textit{Device 1} is executing a third party application activity \textit{(a client)}  which contains a valid clickable URL link. By clicking a link, typically, Android applications generate an implicit intent asking to the OS to open and show the page linked by the URL. Usually, in the absence of other applications capable of solving this kind of implicit intent, the process ends with the opening of a browser in the same device, which opens the URL in one of its activities. In this case the Liquid Android application server, installed on both devices, should to tell the OS that it can handle that implicit intent, find the other devices in the net, in this case the so called \textit{Device 2}, le the user choose with which device complete the task, convert the intent in a network message and send it to that device. Once the message arrived to the \textit{Device 2} the Liquid Android middleware server application is responsible to translate again the received message into the starting implicit intent sent by the \textit{Device 1} and to start the very same resolution mechanism for that intent by its own Android operating system, which should end by the opening of a browser activity to view the page.\\
It is clear that a solution for this scenario would be also, once implemented, a solution for the second problematic scenario proposed in Section \ref{devAPI}. In fact we can consider the development of an API to build native Android distributed systems as a sub problem of the first one, already described above and in Section \ref{middlewarescenario}. The implemented version of the general solution could be also used as a library to implement special purposes similar systems by simply extending my framework and including its implemented Java classes in other Android applications projects. For these reasons in developing the solution I will try to make it as clear as possible, and to parametrize as far as possible the settings variable of my framework to make it easily extensible and ready to use buy other Android developers.\\
I would like now to list the goals that my work has to match, in order to be a valid proposal for solving the given problem. These goals are not to be intended as set in stone, they are the general motivation that leads to construct a prototype of the proposed software architecture. According to my thoughts during the development, it is possible to identify the following goals:
\begin{itemize}
	\item The middleware must work without any proprietary application: it has to interface itself to the upper layer without installing any other application of any vendor in the owned device. It must be completely neutral to the market, it must work with any version of the Android operating system starting from the API level 19, also with Android customized versions developed by device maker like Samsung, LG, Huawei and any other brand. It is the fundamental requirement to create heterogeneous applications and to separate the various closed solutions of today and an open solution for everyone in the future.
	\item The middleware has to simplify the life of the developer, he should not have to worry too much about the substrates, he should be able to fast prototype. The developer should see my framework as help for his work. The idea is to provide a ready to use service, with which it is possible create new application by exploiting it.
	\item The middleware should offer the user the possibility to access directly to other devices in the network without the need of configure anything. Users, once installed the middleware should use its functionalities of receiving/forwarding intents in a transparent way, in the same way they use other applications and with the same mechanism they learned by using the standard Android operating system.
\end{itemize}
The next sections contain all the steps necessary to have a full working system.
Firstly I would solve the, let me call it \textit{general theoretical problem} by dividing it as discussed above and providing the solution for each of them, taking into account also the data management scenario. Then I would like to present the structure of the development API, while the actual implementation of the working Liquid Android application is left for the next chapter with some working tests and a deep component analysis.


\section{Proposed Solution}
\par In this section i will perform an in depth analysis of the possible solution to the given problem: how I can extend the Android OS providing it distributed functionalities.
\subsection{Network Architecture}
The first step while creating a distributed system is to define the networking architecture, in particular I must define the kind of nodes involved in the system and the way in which they interact, what they can do, which operation they can perform and in which way. As mentioned earlier the network architecture of the system must fulfill the following requirements:
\begin{itemize}
 \item \textit{dynamicity}, it must allow any device to perform the dynamic connection, and also disconnection, to the distributed system at any time, since the nodes of the network are mainly mobile devices and they can be moved easily. Nodes can \textit{JOIN} and \textit{LEAVE} every time, and the network must accommodate them automatically.
  \item \textit{simplicity}, the network must be as simple as possible, it should not need any particular configuration on the nodes to \textit{JOIN}. Any node should perform other nodes \textit{DISCOVERY} in the network in a easy way without the need to know them a priori.
 \item \textit{reliability and security}, are important non functional requirements in such a system. I want to make the network reliable and secure as much as possible by the adoption of standard software engineering techniques.
\end{itemize}
The network architecture that fits better all the requirements listed above is certainly a P2P network. As seen in \ref{P2P}, in peer-ro-peer networks there is not a clear distinction between clients and servers, in fact a peer-to-peer network is designed around the notion of equal peer nodes simultaneously functioning as both "clients" and "servers" to the other nodes on the network.

\begin{figure}[h]
\centering
\includegraphics[width=.9\textwidth]{P2Pnetwork}
\caption{P2P Liquid Android network example}
\label{fig:4.3}
\end{figure}
I think that an \textit{unstructured P2P architecture} is the best choice for my system due to the fact that it do not impose a particular structure on the overlay network by design and data is still exchanged directly over the underlying TCP/IP network, so at the application layer peers are able to communicate with each other directly, via the logical overlay links. In \figurename~\ref{fig:4.3} there is an example of my architecture in which different devices are in range under the same local network and they can communicate sending intents and data.\\
To obtain this kind of structure, and let it change dynamically depending on the devices in range, I need to equip the devices with a \textit{network service}. In computer networking, a network service is an application running at the network application layer and above, that provides data storage, manipulation, presentation, communication or other capability which, in this case will be used in combination with a peer-to-peer architecture based on the application layer network protocols. Different services use different packet transmission techniques. In general, packets that must get through in the correct order, without loss, use TCP, whereas real time services where later packets are more important than older packets use UDP. For example, file transfer requires complete accuracy and so is normally done using TCP, and audio conferencing is frequently done via UDP, where momentary glitches may not be noticed. In this case I will adopt the TCP transport layer because I need to transfer packets in reliable way, as much as possible, and also avoid network congestions, the UDP protocol in fact lacks built-in network congestion avoidance while TCP has it.\\
To fulfill the above listed requirements of dynamicity and simplicity the right approach should be the Zeroconf one, already discussed in \ref{zeroconf}. Using a Zeroconf implementation to register and discover the service in the LAN will let the node to connect dynamically and in a easy way to the distributed system. The three main operation a node can perform are:
\begin{itemize}
	\item \textit{JOIN}, to join the system a node must activate the network service, it must provide an internal endpoint for sending or receiving data in the computer network. The best abstraction to do that is to open a socket, as seen in \ref{socket}. In particular, a TCP socket is characterized by two main parameters: the \textit{IP} and the service \textit{PORT}. In my system the TCP socket is a great choice because, as already stated, it is a network abstraction and it can be used to let communicate heterogeneous devices and can be implemented in several different ways using basically any development language. By knowing the couple of variable of the network service another device can send message, streams and so on, to it through the TCP socket.  Since mobile devices changes frequently their connections variables, because of their nature to be easily moved from one place to another, we need a system that can identify them dynamically. The name of the service and the chosen transport layer can be established once for all and they can never change. In this environment Zeroconf provides service registration and discovery. By registering the service name, port, and transport layer the node can be found in the LAN by other nodes looking for that kind of service. In this way nodes do not need to know, a priori, the two variables, IP and PORT of any node in the network, to communicate with each other, they, indeed, need only to know the service name and the chosen transport layer to find other nodes in the network.
	\lstinputlisting[language=Java , caption=Zerconf registration example , label=code:4.1]{Codici/registerservice.java}
	The snippet of code \ref{code:4.1} is an example of how in Android it is possible to register a service using the Zeroconf approach. Once registered the service Zeroconf provides name resolution functionalities to discover other nodes and then connect to them to start the communication.
	\item \textit{LEAVE}, I want that a node can decide whether join or leave the network at any time. To leave the network a node should only unregister the service registered using Zeroconf and then close the socket to avoid accidental or malicious connections.
	\item \textit{SEARCH}, since the network is dynamic, it is necessary to determine how the nodes, once the service has registered, can search for and find other nodes. As already mentioned describing the \textit{JOIN} operation Zeroconf also provide the network service discovery and the naming resolution mechanism. In this case Zeroconf uses a \textit{Domain Name System (DNS) based Service Discovery}, the so called \textit{DNS-DS}. DNS-SD allows clients to discover a named list of service instances, given a service type, and to resolve those services to \textit{hostnames} using standard DNS queries. The specification is compatible with existing \textit{unicast DNS} server and client software, but works equally well with \textit{multicast DNS (mDNS)} in a zero-configuration environment. Each service instance is described using a \textit{DNS SRV} and \textit{DNS TXT} record. A client discovers the list of available instances for a given service type by querying the \textit{DNS PTR} record of that service type's name; the server returns zero or more names of the form "<Service>.<Domain>", each corresponding to a SRV/TXT record pair. The SRV record resolves to the domain name providing the instance, while the TXT can contain service-specific configuration parameter. Once completed the resolution process the node which started it to find other nodes in the network, knows any couple of IP/PORT of their open sockets and can connect to them to exchange messages or transfer data.
\end{itemize}
Given the network structure security is obviously a non functional requirement in building my system, but since my entire middleware, once connected, will allow to send and execute any kind of task, in the form of implicit intents, to each device involved I need to make some considerations about the security of such a system. The security of my system is highly influenced by the underlying physical network, if the LAN access is secured and protected my system will be, indeed, enough secure. The network should be protected using a firewall and the service ports used by the device involved in the system should not be accessible from the outside of the LAN. Moreover the WiFi should be protected using a strong password and a secure and updated access protocol like the\textit{WPA2}. Furthermore, I want that users of my system are aware of the fact that it is running, so once the service is activated users will be alerted by a notification in the appropriate Android notification area.
 
 \subsection{Communication Model}
 Once the network is up the second step is let the device communicate to cooperate giving, to the final users, the feeling that they are working with a single big distributed operating system. To achieve this goal, as already explained several times, my system is supposed to intercept implicit intents, let the user select a target device, or a group of them, to perform the task, send it and once arrived perform it in the selected device or group devices. With TCP sockets it very easy to exchange messages between networked devices, so it is equally easy to understand why I have chosen a message oriented structure as the communication model, briefly presented in \ref{messageo}, for my system. Furthermore, as already stated with the \tablename~\ref{tab:comp} such a model gives me more freedom of choice regarding certain characteristics that the system must have, I want the communication is:
 \begin{itemize}
	\item \textit{concurrent}, in the original Android OS when an intent resolution mechanism is started by any other component of the system, for example an application activity, the OS stops its execution to perform, in foreground, the task the intent contains. I want to leave unchanged this kind of mechanism in my system, so when a, let me call it, \textit{distributed intent} arrives from the socket, the operating system must treat it exactly as if it were a \textit{local intent} and execute it in the same way it does with other intents. The arrival of a new distributed intent from another device is the event which triggers the standard OS resolution technique by suspending any other operation. I want the network listen and accepts messages with distributed intents to solve at any moment, in a fully concurrent way, so the last intent is triggered the first it is resolved and executed in a \textit{LIFO}, last in first out, way like in a stack structure.
	\item \textit{asynchronous}, the standard intent resolution mechanism, like many other in the Android operating system, is mainly asynchronous. When an implicit intent is triggered it will be performed without knowing, a priori, what application will execute the task and how. To be more clear, for example when an application ask the OS to open a map of a place, the application which started the process do no need to know if the system completed the task and with which results. I want that my system do not need to change this behavior, if an operation need to be performed the intent is triggered and the OS reacts as usual without the need to send back acknowledgment messages \text{ACKs}. When it is necessary to synchronize the communication among the various components involved when an implicit intent is triggered Android provides a mechanism, using the system function \textit{startActivityForResult(intent)}, to send the result back to the caller. I intend to keep the same mechanism also when dealing with distributed intents, but I want to maintain an asynchronous approach also in this case, the caller could continue its execution without waiting for the results and then, once done, receive back the response, in the form of another distributed intent, from the called. I will give further details of this case when presenting the kind of messages can be sent using my system and the data model solution part.
 \end{itemize}

Now the difficult part is establish how intents can be sent through the sockets and which kind of messages my system can send and then handle to perform various kind of tasks.\\
The next step, hence, is to find a way to send the intents from a device to another without loosing information and once they are arrived to be executed like they were local intents. To do this job it is necessary to analyze in dept what an intent is in Android and what it can contain.\\
As showed in the chapter \ref{cap:statoarte} in \ref{intents} intents are the way in which standard Android framework's components communicate among each other, and they also represent the abstraction of actions to be performed by Android applications. In the Android developer framework an intent is implemented as a Java Class containing the information and data to perform the task. An intent is an abstract description of an operation to be performed. It can be used in various ways: with \textit{startActivity} to launch an Activity, with \textit{broadcastIntent} to send it to any interested \textit{BroadcastReceiver} components, and  with \textit{startService(Intent)} or \textit{bindService(Intent, ServiceConnection, int)} to communicate with a background Service. However its most significant use is in the launching of activities, where it can be thought of as the glue between activities. It is basically a passive data structure holding an abstract description of an action to be performed \cite{android2017intent}.\\
Every implicit intent is characterized, mainly, by an \textit{ACTION}, \textit{DATA} and a bundle of \textit{EXTRAS}.
\begin{table}[h]
	%
	\caption{Intent Structure}
	%
	\label{tab:intent}
	%
	\centering
	\begin{center}
	
	%
	\begin{tabular}{>{\centering\arraybackslash} m{0.15\textwidth}p{0.4\textwidth}p{0.4\textwidth}}
		%
		
		\toprule
		%
		\centering\textbf{Attribute} & \centering\textbf{Description}  &\begin{minipage}[t]{0.4\textwidth}
			\centering
		\textbf{Examples}
		\end{minipage} \\
		%
		\midrule
		%
		\centering\textit{ACTION} & \begin{minipage}[t]{0.4\textwidth}
			\centering
		The general action to be\\ performed.
		\end{minipage} & \begin{minipage}[t]{0.4\textwidth}
		\centering
		ACTION\_VIEW\\ACTION\_EDIT
		\end{minipage}\\ %fine riga tabella
		&&\\
		\centering\textit{DATA} & \begin{minipage}[t]{0.4\textwidth}
			\centering
		The data to operate on,\\expressed as a Uri.
		\end{minipage} &
		\begin{minipage}[t]{0.4\textwidth}
			\centering
		content://contacts/people/1 \\ tel:123
		\end{minipage}\\  %fine riga tabella
		&&\\
		\centering\textit{CATEGORY} & \begin{minipage}[t]{0.4\textwidth}
			\centering
			Gives additional information\\about the action to execute.
		\end{minipage} &
		\begin{minipage}[t]{0.4\textwidth}
			\centering
			 CATEGORY\_LAUNCHER \\  CATEGORY\_ALTERNATIVE
		\end{minipage}\\  %fine riga tabella
		&&\\
		\centering\textit{TYPE} & \begin{minipage}[t]{0.4\textwidth}
			\centering
			Specifies an explicit type \\(a MIME type) of the intent data. Normally the type is inferred\\ from the data itself. 
		\end{minipage} &
		\begin{minipage}[t]{0.4\textwidth}
			\centering
			type */*
		\end{minipage}\\  %fine riga tabella
		&&\\
		\centering\textit{EXTRAS} & \begin{minipage}[t]{0.4\textwidth}
			\centering
		This is a Bundle of any additional information. This can be used to provide extended information to the component.
		\end{minipage} &
		\begin{minipage}[t]{0.4\textwidth}
			\centering
			EXTRA\_TEXT\\
			EXTRA\_TITLE
		\end{minipage}\\  %fine riga tabella
	
		\bottomrule
		%
	\end{tabular}
\end{center}
	%
\end{table}
The \tablename~\ref{tab:intent} explains in depth what an intent could contain, of course there are many types of action, data, categories and extra that are not reported here for reasons of brevity but that can be found in \cite{android2017intent}. The real problem, for the purposes of my system, is that the Android intent Java Class can no be serialized,automatically and send through the socket as it is. In computer science serialization is the process of translating data structures or object state into a format that can be stored, or transmitted across a network connection link, and reconstructed later in the same or another computer environment.\\ When the resulting series of bits is reread according to the serialization format, it can be used to create a semantically identical clone of the original object \cite{marshall2015c++}. In a Java environment, like the Android one, Java Classes that implements the \textit{Serializable interface} can be automatically serialized, by the environment itself, and sent through a socket  and then automatically deserialized to reconstruct the original sent object by the receiver, but as already told the intent Class does not implement this functionality. I need therefore to find a way to do this process in a fully functional and also efficient way. There are, actually, many alternatives to accomplish this objective. For example I could generate a new Java Object, containing all the intent attributes, that implements the Java Serializable interface, convert intents to this new Object and then let Java perform its automatic serialization/serialization. As an alternative I could generate some kind of semi-structured data, using a well know, let me call it \textit{container language}, such as JSON or XML, which can be easily sent, as a string, over the socket and then parsed to rebuild the original Java Object.\\
Since the effort required to develop one of the two solution, presented above as examples, is practically the same, I decided to opt for the second alternative, using a JSON structure, because it has substantial advantages.\\
JSON (JavaScript Object Notation) is a lightweight data-interchange format. It is easy for humans to read and write. It is easy for machines to parse and generate. It is based on a subset of the JavaScript Programming Language, it is completely language independent. These properties make JSON an ideal data-interchange
language \cite{w3c2007introducing}.\\
It is based mainly on two types of structures:
\begin{itemize}
	\item Key/Value pair set: it can be considered an object of an Object Oriented programming language,
	\item Collection of elements: it can be considered an array of Objects.
\end{itemize}
By converting any possible implicit intent in a \textit{JSON-intent} the system complies with the following properties:
\begin{itemize}
	\item \textit{M2M communication}, and also \textit{M2H communication}, as already stated JSON are really easy to read and understand by human and also by machines,
	\item \textit{lightweight messages} and also \textit{small overhead}, it is well known that Serialization in Java produce a big overhead and serialization/deserialization to be more general lacks efficiency. JSON are merely \textit{well formatted Strings} which can be sent unaltered on the socket without the need be further converted. The sender send the JSON as a string and the receiver gets exactly that string.
	\item \textit{freedom}, JSON is optimized to be used with a great variety of programming languages, the creation of a \textit{JSON-intent} opens up various possibilities for further development: using the correct syntax, any device, even non-Android, could be a client of my system. In this way it would be possible to send original Android intents to any Android device, with my service installed and active, from a Client running in different OS and using different programming languages. It is possible, in fact, to use Zeroconf to find the service in many other environment and to connect to the socket to send JSON-intents to the connected Android devices.
\end{itemize}
To obtain a general solution to convert any possible Android intent in to a JSON-intent, and viceversa, i need to define the syntax of new object which must fit as much as possible to the structure of the Android intent Class. In fact it would be very easy to convert the major fields that characterize a given intent that are mostly correctly formatted strings, if it were not for the fact that the bundle of extras can contain different types of structured data.
\subsubsection{JSON-Intent sintax}\label{syntax}
In this subsection I am going to define step by step the syntax of my solution, explaining with listings and images all the elements.
An important and remarkable reflection is needed here: unfortunately it is not possible for me alone to consider every kind of intents and to convert properly any kind of data, but my system will work with great majority of them. It is in fact unlikely that implicit intents will contain strange structured data types or \textit{Parcelables and Serializable} data which are usually used with implicit intents, which, for construction of the operating system, my system can not handle. For these reason I am not going to support, with this solution, that kind of data, since it would be very difficult, and highly inefficient, send and reconstruct Parcelables and Serializable data types over my system. That said, I want to proceed with the definition of the syntax of the JSON-Intents.\\
\begin{figure}[h]
	\centering
	\includegraphics[width=.9\textwidth]{intentUML}
	\caption{UML  structure for an Intent}
	\label{fig:4.4}
\end{figure}
In \figurename~\ref{fig:4.4} it is shown the complete structure of an intent including the detailed description, using an UML model, of the bundle that is responsible of holding the Extras. The bundle is a sort of container of a non predetermined number of couples \textit{<Key,Value>} in which the Key is always an arbitrary String, while the Value can be any of the structured data types listed in the schema.
The \textit{"*"} notation at the end of any attribute in the bundle, means precisely that the number of \textit{<Key,Value>} is variable, and optional, for any kind of structured Value data type.\\
I have already analyzed the structure of an Android intent object in the \tablename~\ref{tab:intent}, now I am reproducing the same structure to convert it in a JSON-intent. Using some tables i would explain all the syntax i have created. In the \tablename~\ref{tab:JSON-Intent} it is shown the general structure for the JSON-Intent, precisely in the first column, \textit{Name}, there is in bold the String keyword used in JSON to identify the field. In the second column, \textit{Type}, it is explained which kind of data type the field, identified by the keyword in the first column, contains. Since JSON allows to store only \textit{String},	\textit{Boolean}, \textit{JSON Array}, \textit{Number} and \textit{JSON Object} data types in the third column, \textit{description}, there is a short explanation of how the conversion is performed.
\bigskip
\bigskip
\bigskip
\bigskip
\begin{table}[h]
	%
	\caption{JSON-Intent fields}
	%
	\label{tab:JSON-Intent}
	%
	\centering
	\begin{center}
		
		%
		\begin{tabular}{>{\centering\arraybackslash} m{0.15\textwidth}p{0.25\textwidth}p{0.55\textwidth}}
			%
			
			\toprule
			%
			\centering\textbf{Name} & \centering\textbf{Type}  &	\textbf{Description} \\
			%
			\midrule
			%
			\centering\textbf{action} & \begin{minipage}[t]{0.25\textwidth}
				\centering
				String
			\end{minipage} & \begin{minipage}[t]{0.55\textwidth}
				It was a String also in the original Intent Object and do not need to be further converted. It is used also to identify, by using intent filters, which applications can execute the intent once arrived to the receiver.
			\end{minipage}\\%fine riga tabella
		&&\\
			\centering\textbf{[data]}\footnotemark[1] & \begin{minipage}[t]{0.25\textwidth}
				\centering
				String
			\end{minipage} & \begin{minipage}[t]{0.55\textwidth}
				It was a URI type in the Intent Object which can be easily stored as a string by using the URI.toString() method in Java.
			\end{minipage}\\%fine riga tabella
		&&\\
			\centering\textbf{[categories]} & \begin{minipage}[t]{0.25\textwidth}
			\centering
			Array\\<Category> \footnotemark[2]
		\end{minipage} & \begin{minipage}[t]{0.55\textwidth}
			It was a Set of Strings in the Intent Object so it can be converted in an array of Category JSON Objects \textit{(JSON Array)} containing all the category of the intent.
		\end{minipage}\\%fine riga tabella
		&&\\
			\centering\textbf{[type]}\footnotemark[1] & \begin{minipage}[t]{0.25\textwidth}
			\centering
			String
		\end{minipage} & \begin{minipage}[t]{0.55\textwidth}
			It was a String also in the original Intent Object and do not need to be further converted.
		\end{minipage}\\%fine riga tabella
		&&\\
			\centering\textbf{[extras]}\footnotemark[1] & \begin{minipage}[t]{0.25\textwidth}
			\centering
			Array\\<Bundle Object>\footnotemark[2]
		\end{minipage} & \begin{minipage}[t]{0.55\textwidth}
			It is a collection of all possible extras, it has to be converted in complex JSON Object which I will describe in the next tables.
		\end{minipage}\\%fine riga tabella	
			\bottomrule
			%
		\end{tabular}
	\end{center}
	%
\end{table}
\footnotetext[1]{The notation [Field] means that the field is optional}
\footnotetext[2]{The notation Array<Type> means that the Array contains Objects of that type}

Since \textit{categories} and, as anticipated, \textit{extras} are complex fields, containing array data types, introduced by me, I need to give further explanations of my choices in translating this objects form Java to JSON.

\begin{table}[h]
	%
	\caption{Category fields}
	%
	\label{tab:category}
	%
	\centering
	\begin{center}
		
		%
		\begin{tabular}{>{\centering\arraybackslash} m{0.15\textwidth}p{0.25\textwidth}p{0.55\textwidth}}
			%
			
			\toprule
			%
			\centering\textbf{Name} & \centering\textbf{Type}  &	\textbf{Description} \\
			%
			\midrule
			%
			\centering\textbf{category} & \begin{minipage}[t]{0.25\textwidth}
				\centering
				String
			\end{minipage} & \begin{minipage}[t]{0.55\textwidth}
				It was a String also in the original Intent Object and do not need to be further converted.
			\end{minipage}\\%fine riga tabella		
			\bottomrule
			%
		\end{tabular}
	\end{center}
	%
\end{table}

In the \tablename~\ref{tab:category}, using the same notation of the previous table, it is shown how the \textit{Category JSON object} is structured. It is, merely, a simple object containing a single field using the keyword \textit{category} with a String value which is any possible Android intent's category.
\bigskip
\bigskip
\bigskip
\bigskip
\bigskip

\begin{table}[h]
	%
	\caption{JSON-Bundle Object fields}
	%
	\label{tab:Bundle}
	%
	\centering
	\begin{center}
		
		%
		\begin{tabular}{>{\centering\arraybackslash} m{0.15\textwidth}p{0.25\textwidth}p{0.55\textwidth}}
			%
			
			\toprule
			%
			\centering\textbf{Name} & \centering\textbf{Type}  &	\textbf{Description} \\
			%
			\midrule
				%
			\centering\textbf{[key]\footnotemark[1]} & \begin{minipage}[t]{0.25\textwidth}
				\centering
				String
			\end{minipage} & \begin{minipage}[t]{0.55\textwidth}
				It contains the keyword, an arbitrary string, used to store the specific extra in the original Java intent object.
			\end{minipage}\\%fine riga tabella		
			&&\\
			%
			\centering\textbf{type} & \begin{minipage}[t]{0.25\textwidth}
				\centering
				String
			\end{minipage} & \begin{minipage}[t]{0.55\textwidth}
				It is a string which describes the type of structured Java data type, it is used to correctly reconstruct the original Intent Object.
			\end{minipage}\\%fine riga tabella		
			&&\\
			\centering\textbf{data} & \begin{minipage}[t]{0.25\textwidth}
				\centering
				Variable
			\end{minipage} & \begin{minipage}[t]{0.55\textwidth}
				This filed is the one which contains the actual data, it can be a primitive data type such as a boolean a double and so and, or a supported array. The full description of this field is given in \tablename~\ref{tab:data}.
			\end{minipage}\\%fine riga tabella		
			\bottomrule
			
			%
		\end{tabular}
	\end{center}
	%
\end{table}
In the same way the \tablename~\ref{tab:Bundle} shows the structure of the JSON-Bundle object. Every single JSON-Bundle object represent a possible \textit{extra} of the original Java intent. The field \textit{key} contains the string keyword used to identify the extra. The \textit{type} field contains one of the keywords, in bold, defined in the next table, which is a String used to identify the structured Java data type, used to perform the so called \textit{deserialization}, by reconverting the JSON-Intent in a Android Java intent object. Depending on the type of data that it is necessary to store the \textit{data} field can contain the actual data, if the original Java type was a primitive data type, or an array of JSON-Bundle objects without the field \textit{key}, if the original data type was an Array or an ArrayList.\\

	%
	

		%
		\begin{longtable}{|p{0.19\textwidth}|p{0.17\textwidth}p{0.17\textwidth}p{0.37\textwidth}|}
			%
			
			\toprule
			%
			\centering\textbf{Java Object} & \multicolumn{3}{ |c| }{\textbf{JSON Object}}\\
		
			%
			\midrule
			\centering\textbf{name}	& \centering\textbf{type content}  &	\centering\textbf{data type} &	\textbf{description} \\ \hline
			\multicolumn{4}{ |c| }{ }\\
			\multicolumn{4}{ |c| }{\textbf{Primitive Data Types}}\\ \hline
			&&&\\
			%
			\centering boolean & \begin{minipage}[t]{0.17\textwidth}
				\centering
				\textbf{boolean}
			\end{minipage} & \begin{minipage}[t]{0.17\textwidth}
			\centering
				Boolean
			\end{minipage} & \begin{minipage}[t]{0.37\textwidth}
			The data filed contains the actual boolean value.
			\end{minipage}\\%fine riga tabella		
			&&&\\
			%
			\centering byte & \begin{minipage}[t]{0.17\textwidth}
				\centering
				\textbf{byte}
			\end{minipage} & \begin{minipage}[t]{0.17\textwidth}
				\centering
				Number
			\end{minipage} & \begin{minipage}[t]{0.37\textwidth}
				The data filed contains the actual byte value.
			\end{minipage}\\%fine riga tabella		
			&&&\\
			%
			\centering CharSequence & \begin{minipage}[t]{0.17\textwidth}
				\centering
				\textbf{charsequence}
			\end{minipage} & \begin{minipage}[t]{0.17\textwidth}
				\centering
				String
			\end{minipage} & \begin{minipage}[t]{0.37\textwidth}
				The data filed contains the actual CharSequence value converted in String.
			\end{minipage}\\%fine riga tabella		
			&&&\\
			%
			\centering double & \begin{minipage}[t]{0.17\textwidth}
				\centering
				\textbf{double}
			\end{minipage} & \begin{minipage}[t]{0.17\textwidth}
				\centering
				Number
			\end{minipage} & \begin{minipage}[t]{0.37\textwidth}
				The data filed contains the actual double value.
			\end{minipage}\\%fine riga tabella		
			&&&\\
			%
			\centering float & \begin{minipage}[t]{0.17\textwidth}
				\centering
				\textbf{float}
			\end{minipage} & \begin{minipage}[t]{0.17\textwidth}
				\centering
				Number
			\end{minipage} & \begin{minipage}[t]{0.37\textwidth}
				The data filed contains the actual float value.
			\end{minipage}\\%fine riga tabella		
			&&&\\
			%
			\centering int & \begin{minipage}[t]{0.17\textwidth}
				\centering
				\textbf{integer}
			\end{minipage} & \begin{minipage}[t]{0.17\textwidth}
				\centering
				Number
			\end{minipage} & \begin{minipage}[t]{0.37\textwidth}
				The data filed contains the actual int value.
			\end{minipage}\\%fine riga tabella		
			&&&\\
			%
			\centering long & \begin{minipage}[t]{0.17\textwidth}
				\centering
				\textbf{long}
			\end{minipage} & \begin{minipage}[t]{0.17\textwidth}
				\centering
				Number
			\end{minipage} & \begin{minipage}[t]{0.37\textwidth}
				The data filed contains the actual long value.
			\end{minipage}\\%fine riga tabella		
			&&&\\
			\centering short & \begin{minipage}[t]{0.17\textwidth}
				\centering
				\textbf{short}
			\end{minipage} & \begin{minipage}[t]{0.17\textwidth}
				\centering
				Number
			\end{minipage} & \begin{minipage}[t]{0.37\textwidth}
				The data filed contains the actual short value.
			\end{minipage}\\%fine riga tabella		
			&&&\\
			\centering String & \begin{minipage}[t]{0.17\textwidth}
				\centering
				\textbf{string}
			\end{minipage} & \begin{minipage}[t]{0.17\textwidth}
				\centering
				String
			\end{minipage} & \begin{minipage}[t]{0.37\textwidth}
				The data filed contains the actual String value.
			\end{minipage}\\%fine riga tabella		
			 \hline
			\multicolumn{4}{ |c| }{ }\\
			\multicolumn{4}{ |c| }{\textbf{Array Data Types}}\\ \hline
			% inizio array
			&&&\\
			\centering boolean[] & \begin{minipage}[t]{0.17\textwidth}
				\centering
				\textbf{aboolean}
			\end{minipage} & \begin{minipage}[t]{0.17\textwidth}
				\centering
				Array\\<Bundle Object>\footnotemark[2]
			\end{minipage} & \begin{minipage}[t]{0.37\textwidth}
				The data filed contains an JSON Array of Bundle Objects having in the type field \textit{"boolean"} and in the data filed the actual boolean value.
			\end{minipage}\\%fine riga tabella		
			&&&\\
			%
			\centering byte[] & \begin{minipage}[t]{0.17\textwidth}
				\centering
				\textbf{abyte}
			\end{minipage} & \begin{minipage}[t]{0.17\textwidth}
				\centering
				Array\\<Bundle Object>\footnotemark[2]
			\end{minipage} & \begin{minipage}[t]{0.37\textwidth}
				The data filed contains an JSON Array of Bundle Objects having in the type field \textit{"byte"} and in the data filed the actual byte value.
			\end{minipage}\\%fine riga tabella		
			&&&\\
			%
			\centering CharSequence[] & \begin{minipage}[t]{0.17\textwidth}
				\centering
				\textbf{acharsequence}
			\end{minipage} & \begin{minipage}[t]{0.17\textwidth}
				\centering
				Array\\<Bundle Object>\footnotemark[2]
			\end{minipage} & \begin{minipage}[t]{0.37\textwidth}
				The data filed contains an JSON Array of Bundle Objects having in the type field \textit{"charsequence"} and in the data filed the actual charsequence value.
			\end{minipage}\\%fine riga tabella		
			&&&\\
			%
			\centering double[] & \begin{minipage}[t]{0.17\textwidth}
				\centering
				\textbf{adouble}
			\end{minipage} & \begin{minipage}[t]{0.17\textwidth}
				\centering
				Array\\<Bundle Object>\footnotemark[2]
			\end{minipage} & \begin{minipage}[t]{0.37\textwidth}
				The data filed contains an JSON Array of Bundle Objects having in the type field \textit{"double"} and in the data filed the actual double value.
			\end{minipage}\\%fine riga tabella		
			&&&\\
			%
			\centering float[] & \begin{minipage}[t]{0.17\textwidth}
				\centering
				\textbf{afloat}
			\end{minipage} & \begin{minipage}[t]{0.17\textwidth}
				\centering
				Array\\<Bundle Object>\footnotemark[2]
			\end{minipage} & \begin{minipage}[t]{0.37\textwidth}
				The data filed contains an JSON Array of Bundle Objects having in the type field \textit{"float"} and in the data filed the actual float value.
			\end{minipage}\\%fine riga tabella		
			&&&\\
			%
			\centering int[] & \begin{minipage}[t]{0.17\textwidth}
				\centering
				\textbf{ainteger}
			\end{minipage} & \begin{minipage}[t]{0.17\textwidth}
				\centering
				Array\\<Bundle Object>\footnotemark[2]
			\end{minipage} & \begin{minipage}[t]{0.37\textwidth}
				The data filed contains an JSON Array of Bundle Objects having in the type field \textit{"integer"} and in the data filed the actual int value.
			\end{minipage}\\%fine riga tabella		
			&&&\\
			%
			\centering long[] & \begin{minipage}[t]{0.17\textwidth}
				\centering
				\textbf{along}
			\end{minipage} & \begin{minipage}[t]{0.17\textwidth}
				\centering
				Array\\<Bundle Object>\footnotemark[2]
			\end{minipage} & \begin{minipage}[t]{0.37\textwidth}
				The data filed contains an JSON Array of Bundle Objects having in the type field \textit{"long"} and in the data filed the actual long value.
			\end{minipage}\\%fine riga tabella		
			&&&\\
			\centering short[] & \begin{minipage}[t]{0.17\textwidth}
				\centering
				\textbf{ashort}
			\end{minipage} & \begin{minipage}[t]{0.17\textwidth}
				\centering
				Array\\<Bundle Object>\footnotemark[2]
			\end{minipage} & \begin{minipage}[t]{0.37\textwidth}
				The data filed contains an JSON Array of Bundle Objects having in the type field \textit{"short"} and in the data filed the actual short value.
			\end{minipage}\\%fine riga tabella		
			&&&\\
			\centering String[] & \begin{minipage}[t]{0.17\textwidth}
				\centering
				\textbf{astring}
			\end{minipage} & \begin{minipage}[t]{0.17\textwidth}
				\centering
				Array\\<Bundle Object>\footnotemark[2]
			\end{minipage} & \begin{minipage}[t]{0.37\textwidth}
				The data filed contains an JSON Array of Bundle Objects having in the type field \textit{"string"} and in the data filed the actual string value.
			\end{minipage}\\%fine riga tabella
			 \hline
			\multicolumn{4}{ |c| }{ }\\
			\multicolumn{4}{ |c| }{\textbf{ArrayList Data Types}}\\ \hline
			% inizio arraylist
			&&&\\
			%
			\begin{minipage}[t]{0.19\textwidth}
				\centering
				 ArrayList\\<CharSequence>
			\end{minipage} & \begin{minipage}[t]{0.17\textwidth}
				\centering
				\textbf{alcharsequence}
			\end{minipage} & \begin{minipage}[t]{0.17\textwidth}
				\centering
				Array\\<Bundle Object>\footnotemark[2]
			\end{minipage} & \begin{minipage}[t]{0.37\textwidth}
				The data filed contains an JSON Array of Bundle Objects having in the type field \textit{"charsequence"} and in the data filed the actual charsequence value.
			\end{minipage}\\%fine riga tabella		
			&&&\\
			\begin{minipage}[t]{0.19\textwidth}
				\centering
				ArrayList\\<Integer>
			\end{minipage} & \begin{minipage}[t]{0.17\textwidth}
				\centering
				\textbf{alinteger}
			\end{minipage} & \begin{minipage}[t]{0.17\textwidth}
				\centering
				Array\\<Bundle Object>\footnotemark[2]
			\end{minipage} & \begin{minipage}[t]{0.37\textwidth}
				The data filed contains an JSON Array of Bundle Objects having in the type field \textit{"integer"} and in the data filed the actual integer value.
			\end{minipage}\\%fine riga tabella		
			&&&\\
			\begin{minipage}[t]{0.19\textwidth}
				\centering
				ArrayList\\<String>
			\end{minipage} & \begin{minipage}[t]{0.17\textwidth}
				\centering
				\textbf{alstring}
			\end{minipage} & \begin{minipage}[t]{0.17\textwidth}
				\centering
				Array\\<Bundle Object>\footnotemark[2]
			\end{minipage} & \begin{minipage}[t]{0.37\textwidth}
				The data filed contains an JSON Array of Bundle Objects having in the type field \textit{"string"} and in the data filed the actual string value.
			\end{minipage}\\%fine riga tabella
			
			\bottomrule
			%
			
			\caption{Possible data types}
			%
			\label{tab:data}
			%
			\centering
		\end{longtable}

In the \tablename~\ref{tab:data} there is the definition for any Java types of data my translation supports: for every Java type, listed in the first column of the table, is provided the corresponding translation, using the last three columns.\\
In this way I completed the translation of any generic Android implicit intent, which can, easily, be converted in a JSON-Intent by generating a JSON complying with the syntax defined above. The conversion process can be automatized developing an \textit{Intent Converter}, which I will include in the \textit{Liquid Android API}, which i will describe in the following sections.

\subsubsection{JSON-Intent example}
In this small subsection I would like to provide a full example of a JSON-Intent document which represents a theoretical proof of concept. Notice that I did not use all the fields, and obviously all the data types, I have defined in the previous pages: not all of them are
mandatory to have a working system and maybe they can result redundant, and and make the example more difficult to understand.\\
The use case describes a working implicit intent used to ask the Android OS to send an email.\\
In the \lstlistingname~\ref{code:4.2} there is. in Java code, an example of a method which creates an Android implicit Intent to send an email and then asks the OS to resolve that Intent by using an Activity. In the \lstlistingname~\ref{code:4.3} there is exactly the same Intent converted using the syntax proposed above by me. The JSON-Intent thus created can be easily sent through the socket and then reconverted back to the original Java Intent object.
\lstinputlisting[language=Java , caption=Implicit Intent example , label=code:4.2]{Codici/mailintent.java}
\begin{lstinputlisting}[
%	morekeywords={action, itype, category, categories, extras, type, data, key},
	language=JSON,
	caption={Conversion of the Intent in Listing \ref{code:4.2} to JSON-Intent},
	label=code:4.3]
	{Codici/mailintent.json}
\end{lstinputlisting}
\subsubsection{Possible Messages} \label{pm}
Having thus defined the communication model and language, i need now to define the possible kind of messages the system can handle: as mentioned earlier Android intents can be used with different methods to start different kinds of Android components, the are mainly 4 : 
\begin{itemize}
	\item \textit{startActivity(Intent intent)}, ask the system to start an Activity to perform the operation contained in the intent passed as parameter.
	\item \textit{startActivityForResult(Intent intent)}, it performs exactly the same operations of the previous method, farther, it is used to receive a result from the activity when it finishes. The caller activity receives the result as a separate Intent object in the activity's onActivityResult() callback.
	\item \textit{startService(Intent intent)} ask the system to start a Service to perform the operation contained in the intent passed as parameter.
	\item \textit{sendBroadcast(Intent intent)} ask the system to send a broadcast message that any app can receive.
\end{itemize}
Since my system is supposed to work, mainly, by using implicit intents, the most common method it will use will be the first. I want, also, to support the use of the other three methods, so since it is impossible to understand what kind of method it is needed only by analyzing any intent filed, i need to define standard attributes to let my system choose the right Android method to start the received intent. For these reasons my system will threat received intents with the first method unless unless otherwise specified. By including in the JSON-intent a standard extra field it is possible to support any method described above. Including in the extras array the extra with \textbf{key}:\textit{"andorid.intent.extra.LIQUIDMETHOD"}, \textbf{type}:\textit{"string"}, \textbf{data}:\textit{"RESULT" or "SERVICE" or "BROADCAST"} my system will start the received intent with the appropriate method. Furthermore during the development of my system I will analyze special cases of well known intents which need a result back, for example intents asking the OS to take a photo are supposed to receive back the taken picture.
 \subsection{Data management Model}
 The last step is to establish how the system has to manage data created, or manipulated, by sending/resolving distributed intents. I found, basically, two solutions, with both different advantages and disadvantages.
 \subsubsection{Files over socket}\label{fos}
 \begin{figure}[ht]
 	\centering
 	\includegraphics[width=.9\textwidth]{datamodel1}
 	\caption{Files Over the Socket example}
 	\label{fig:4.5}
 \end{figure}
 Since the system exploits the LAN and creates a working P2P network, it is possible to send the generated data directly using it by sending result intent or files over the socket. To continue with the previous example, in which one device of the network ask another one to take a photo, the device which perform the operation and holds the original picture can put it on the socket and send its file or the result intent,properly converted in a JSON-Intent, to the device that sent the request. In this way basically it is possible to realize any type of data management operation only by exploiting the underlying network. This first solution has the advantage of not having to use external services to the local network, indeed it do not need to use the Internet connection. There is however the drawback of having to serialize/deserialize any data to be sent over the socket.
 \subsubsection{Cloud Group}
 \begin{figure}[h]
 	\centering
 	\includegraphics[width=.9\textwidth]{datamodel2}
 	\caption{Cloud Group example}
 	\label{fig:4.6}
 \end{figure}
 It would be possible to easily implement a cloud group, with one of the well known cloud services supported by Android, such as \textit{Google drive}, where devices, while my system is running, can store and download data needed by the distributed intent and generated by executing them. In this way, the system should force to update in a cloud storage any data, such as files, text, numbers, results and so on. The cloud storage, indeed, must guarantee the devices involved to easily access right data. This means that before a user can run the system it is necessary to set up the cloud group, in order to grant access to data only to authorized users. Once the group is up, and permissions are correctly managed, every time a data is produced and stored in the cloud, it would be possible to notify devices in the group, interested in the data generated, and let them download, only by querying the cloud system.
  This second solution has the advantage that can be easily implement using existing cloud systems, but has as drawback the fact of it needs an Internet connection and some security issues because data in that way can be accessed also outside my system.
 
\section{Liquid Android API Library}\label{API}
Once found, and described the so called theoretical solution, which mainly solves the \ref{middlewarescenario} scenario, I started the concrete development of the system, trying to solve the problem and also, to create a system which can be easily extended and used to build other purposes systems.\\
In this section I will describe the \textit{Liquid Android Framework} as a development API, solving the second scenario, \ref{devAPI}, pointing out the main components, methods and configuration variables.
\subsection{General Structure}
\begin{figure}[h!]
	\centering
	\includegraphics[width=1\textwidth]{generaluml}
	\caption{Liquid Android General UML}
	\label{fig:4.7}
	\begin{tabular}{ccc}
		\textcolor{OliveGreen}{$\blacksquare$} Android Service & \textcolor{RoyalBlue}{$\blacksquare$} Android Activity & \textcolor{RoyalPurple}{$\blacksquare$} Standard Java Class\\
	\end{tabular}
\end{figure}
I want to start giving an overview of the general structure of my system. I want to have few, but fully functional, components, and to accomplish the standard design pattern \textit{Model View Controller (MVC)}.\\
In my API the model is represented by the JSON-Intent structure described in the previous sections: intents are the way in which my system communicates, so they are the main kind of data to deal with. Then there are the controller components, which are responsible to manage the data, and to build and maintain the network. Finally there are, also, \textit{User Interface (UI)} components, which are responsible for the interaction with the final user of the system.\\
In \figurename~\ref{fig:4.7} there is the complete structure, by using an \textit{UML Class Diagram} of the framework, in which it is possible to identify the various components by looking at the legend under the picture: components in light-blue are Android activity used to interact with the user, so the are, indeed UI components, while all the other classes are identifiable as controller components. The arrows in figure explain the existing relations among the class components, pointing out the way in which they interact in the system.
\subsection{Controller Components}
First of all, I want to analyze the way in which the controller components build and maintain the network and then how they send/receive JSON-Intent objects and execute them.\\
\subsubsection{NsdHelper}
It is a standard Java Class component, which we can see as a \textit{Helper Class}. It contains all the methods used to register the network service and to search and resolve other devices which are using the system, by finding other services in the LAN which are using the same name and the same transport layer. This class is an implementation of the \textit{NSD Android API} already described in \ref{androidNSD}.
Its main configuration parameters are:
\begin{itemize}
	\item \textit{SERVICE\_TYPE}, it is the string used to identify the network services in the LAN when the NsdHelper perform the register and discover operations. It uses the syntax: \textit{"\_<protocol>.\_<transportlayer>"}. 
	\item \textit{SERVICE\_NAME}, it is the instance name: it is the visible name to other devices on the network. The name is visible to any device on the network that is using NSD to look for local services. The name, also, must be unique for any service on the network, and the NSD library automatically handles conflict resolution.
\end{itemize}
 Since i want to give a flexible development API library it is possible to change this parameter by using the proper setter method or by passing two strings to the NsdHelper constructor. For example is possible to use:
 \begin{itemize}
 	\item \textit{setServiceType("\_<protocol>.\_<transportlayer>")}
 	\item \textit{setServiceName("name")}
 	\item \textit{new NsdHelper("\_<protocol>.\_<transportlayer>","name")} 
 \end{itemize} to change the configuration parameters when initializing the NsdHelper Class component.\\
Its main (public) methods are:
\begin{itemize}
	\item \textit{registerService(int port)}, this method performs the local network service registration, by using the variables described above and the service port passed as a parameter to this method.
	\item \textit{discoverServices()}, this method performs the service discovery and resolution, by scanning the local network and saving found services in a list of resolved devices, containing the connection variables of the other compatible service in the network \textit{IP} and \textit{SERVICE\_PORT}.
	\item \textit{stopDiscovery()}, this is the method used to stop the service discovery mechanism.
\end{itemize}

\subsubsection{Server}
It is a standard Java Class component holding the server functionalities of the system. It is the components which opens the sockets and then waits and listens for invocations by other devices clients. It is responsible for the reconstruction of the intent object by using the received JSON-Intent from client, and also it has to allow the concurrency and let clients connect any time.
To perform this duty my server component uses a multi-thread architecture: every time a client connects it starts a separated \textit{server-thread} to communicate with it. Once received the message the server translates it in a executable Android intent and then execute it by using one of the standard Android methods seen in the previous section when explaining the possible kind of messages.\\
Its main (public) methods are:
\begin{itemize}
	\item \textit{initializeServerSocket()}, this method initialize the \textit{ServerSocket} Java component, by automatically selecting a free port in the LAN, which waits and listens for client connections.
	\item \textit{startServer()}, this method is used to start the server-thread when a client performs an invocation, the started server-thread performs the operations contained in the received JSON-Intent message by reconstructing and passing it to the Android OS to be resolved.
	\item \textit{stopServer()}, this method closes, and destroys, the server-thread when the client invocation is finished to avoid memory leaks when using the system. 
\end{itemize}

\subsubsection{Client}
It is a standard Java Class component holding the client functionalities of the system. It is the components which connects to the  server sockets of the other devices in the network and then sends the message containing the intent to be executed. It is responsible for the conversion of the intent object in a JSON-Intent to be sent on the socket. Once created the message the client send it to the selected server and then closes itself in an asynchronous way, without waiting any ACK message or the result of the invocation.\\
Its main (public) methods is:
\begin{itemize}
	\item \textit{startClient(Intent intent)}, this method send the intent, by translating it in a JSON-Intent, to the selected server. Then the client is automatically closed to avoid possible memory leaks when using the system. 
\end{itemize}
\subsubsection{IntentConverter}
It is another \textit{Helper Class}. It contains all the logic and methods used to perform the conversion of any Android Java Intent object in JSON-Intent objects, and viceversa, using the syntax proposed in the previous chapter. This standard Java Class does not have any variable or parameter, and does not need, also a special constructor. It has only two main Static public methods that can be invoked by using the Class name. \\They are:
\begin{itemize}
	\item \textit{intentToJSON(Intent i)}, this is a Static method which performs the conversion from any Intent object, passed as parameter, to a JSON-Intent, returned by the method as a JSONObject, using the syntax created by me and explained in the previous chapter.
	\item \textit{JSONToIntent(JSONObject j)}, this is a Static method which performs the reconstruction of the Java Intent object, by parsing a well formed JSONObject. It returns a fully working original Android Intent object.
\end{itemize}
\subsubsection{LiquidAndroidService}
This is the Android background working service which really performs the extension of the Android OS giving to it distributed functionalities. When this component is started it is responsible of the initialization of the server, and the registration of the network service in the LAN, by calling the methods of the classes explained above, respectively the Server and the NsdHelper. This is a sort of container for the server side components of the middleware. It is implemented as a foreground working Android service, thus when it is in execution the user is alerted by a notification. A foreground service is a service that the user is actively aware of and is not a candidate for the system to kill when low on memory. A foreground service must provide a notification for the status bar, which is placed under the Ongoing heading. This means that the notification cannot be dismissed unless the service is either stopped or removed from the foreground \cite{devandroidrunning}.
\subsection{UI Components}
I want now to explain how the users can control the distributed intent flow, by using standard Android UI components. Even if my system is supposed to run in background it is also necessary to have few UI components to control the background working mechanisms. In this section I will not provide any screenshot of a real Android application, because in this part I want only to analyze the functionalities included in these components. I will include many real UI screenshots while presenting the Liquid Android application I have developed, using this API, in the next Chapter, \ref{cap:proofofconcept}.
\subsubsection{MainActivity}
This is the UI component responsible for control the entire system. It has four important functionalities:
\begin{itemize}
	\item \textit{Listen to implicit intents}, when an implicit intent needs to be resolved, the Android OS starts the resolution mechanism already described in \figurename~\ref{fig:2.4}.
	\lstinputlisting[language=XML , caption=Intent filter example, label=code:4.4]{Codici/intentfilter.xml}
	It searches for the best activity for the intent by comparing it to intent filters based on three aspects: \textit{Action}, \textit{Data (both URI and data type)} and \textit{Category}. A match is only successful if the actions and categories in the Intent match against the filter. So, to let the Liquid Android API listen for any kind of implicit intent, it must include an activity declared with any possible intent filter to be called when an intent need to be executed.\\
	In \lstlistingname~\ref{code:4.4} there is and example of how an Activity needs to be declared in the manifest to be called by the OS when an implicit intent like the one in the JSON-Intent in the translation example, \ref{code:4.2} is triggered.
	By extending the MainActivity component and adding any kind of intent filter to its declaration in the manifest, it is possible to listen practically to any type of existing implicit intents in the Android framework.
	\item \textit{Find other devices in the network}, the MainActivity provides also this functionality by calling the methods of the NsdHelper class, and then it shows the resolved devices in the network in a ordered list.
	\item \textit{Forward intents to selected devices}, once found the other devices in the network, the MainActivity let the user select on which device, or devices, forward the implicit intent it is managing at the moment. To perform this task the MainActivity starts the Client component and exploits its methods.
	\item \textit{Start/stop the background service}, since when the background service is working the device is exposed to many threat, the user can control it by starting and stopping it every time it is needed by clicking on some special buttons in the MainActivity. 
	
\end{itemize}

\subsubsection{ResultActivity}
This is the UI component which can be used when a received intent needs to send data or results back to the caller. This Activity implements the \textit{onActivityResult()} callback method to manage the result or data produced by the received intent. In my framework API this problem is solved by using the files over the socket approach, described in \ref{fos}, so when the callback is triggered results are sent to the caller through the socket as JSON-Intent messages or serialized files, by using the Client class to send a message to the Server component of the caller. It is necessary to have this activity since it is impossible to call the \textit{startActivityForResult(intent)} method from a background service in Android.

\subsection{Use Cases}
The full API code is available at \href{https://github.com/mola15/LiquidAndroid}{github.com/mola15/LiquidAndroid}, which is a public GitHub repository, so anyone can read, clone use and modify my code for different purposes.\\
As already said several times I will use this API to fully develop the Liquid Android APK which will be a working solution for the first and third problematic scenario, but the Liquid Android API can be used as an Android library in the development of native Android distributed applications. By including my API in any development project it is possible to speed up the process by exploiting my working mechanisms and my methods already developed. Since the API is in Java is just as easy to extend the classes and override some of my methods to accomplish special purposes distributed systems.\\
Exploiting my system would be easy to develop for example:
\begin{itemize}
	\item Distributed Android Computing systems, in which Android devices can share hardware resources to reach a common goal.
	\item Android Computer Cluster, in which Android devices can be seen as node of clusters to perform the same task, controlled and scheduled by software.
	\item Distributed Android File system, by refining the data management model.
\end{itemize}
Furthermore it is possible to adapt my work to accomplish many other purposes, at the end of the next chapter I will present a real, concrete small application, which exploits my system and represent a real use case for my framework.

%
% -----------------------------END--------------------------------- %
%
% ------------------------------------------------------------------------ %
% !TEX encoding = UTF-8 Unicode
% !TEX TS-program = pdflatex
% !TEX root = ../Tesi.tex
% !TeX spellcheck = en_US
% ------------------------------------------------------------------------ %
%
% ------------------------------------------------------------------------ %
% 	CONCLUSIONI
% ------------------------------------------------------------------------ %
%
\chapter{Case Study}
%
\label{cap:proofofconcept}
%
% ------------------------------------------------------------------------%
In this chapter I will describe the real implementation of the system, which is the real solution of the problem faced by this thesis work: how it is possible to extend a mobile operating system, in this case Android, with distributed OS functionalities.\\
This chapter is mainly composed of three parts: the first one is a generic information
section in which the proof of concept is explained in terms of technologies
used, requirements to meet, goals and various technicalities. The second one
is the report of the implementation and development of the application, with
choices and descriptions of what has been done. The third part is a working demo of the
just described system, with live working test cases. It contains screenshots of the
application while it is running and a complete description to explain each case
step by step.
\section{Design Choices}
\subsection{Application Description}
As already specified in the previous chapter my system has been implemented as a standard Android application, which can be installed on any Android device starting from the API level 19, Android 4.4 KitKat. The final APK package contains all the files needed for the system installation, and, once installed, the application performs the extension of the Android OS giving to it distributed functionalities. I will use the complete API described in \ref{API} to implement a background working middleware to distribute implicit intent in a LAN to any Android device with the service installed. In this way every time one of the device, having the \textit{Liquid Android APK}
installed, triggers an implicit intent, my application could intercept and send it to any other device to be resolved and executed. The idea of this prototype is to prove that what I have stated, providing the theoretical solution, can work with a real configuration of Android devices in any LAN. Doing this, the thesis work is somehow \textit{"proved"}: my communication language, defined with a JSON file, is concretely usable and working, not to worry the users about the kind of implicit intent they need to execute in one of the devices in the network. The translation process does not represent an issue for my application, because I have developed an automatic intent translator using the correct syntax proposed by me. I will not develop clients for third party systems, even if I stated that it would be possible, especially in Java environments, but I will implement a simple Android application client, generating some standard implicit intent to perform some test with my system.
\subsection{Requirements}
In this small subsection I want to provide a full list of requirements my application must meet. In order to be considered a solution of the given problem, it must fulfill the constrains listed in \ref{problemconstraints} and also comply with functional and non functional requirements.
\subsubsection{Functional Requirements}
Functional requirements are, indeed, the main functionalities the systems must have in order to properly work to perform desired task.\\
The following lists summarize the main features of the system, so as to ensure a quick reference while reading this document:
\begin{itemize}
	\item \textbf{FR1}: listen to implicit intents.\\My application should declare itself, in the android manifest, as a multipurpose application which can be used to resolve, basically, any kind of implicit intents, in order to be selected by the Android OS whenever an intent resolution process is triggered.
	\item \textbf{FR2}: JSON to Intent, and Intent to JSON, conversion.\\ My application must be able to perform the conversion using the JSON syntax i have explained in \ref{syntax}.
	\item \textbf{FR3}: forward implicit intents.\\ My application must be able to forward any of the implicit intent it can listen, to other LAN connected devices with the \textit{Liquid Android APK} installed.
	\item \textbf{FR4}: receive and execute intents.\\ My application must be able to receive in any moment implicit intents, as JSON-Intent object, and then, let the OS resolve and execute them with its standard mechanisms.
\end{itemize}
\subsubsection{Non-Functional Requirements}
Non-functional requirements are important properties that my
system must have in order to guarantee full functionalities. They are not specific
for my problem but, they are general requirements a system needs to be considered complete. It is quite clear how a system can use my language but if it takes 15 minutes to perform a translation or to deliver a message it is completely useless.\\
Non-functional requirements in this way complete my system, they are mainly:
\begin{itemize}
	\item \textit{Portability}: to have my application used by the largest number of users
	possible, so i have made the choice to use Android API level 19, to allow the installation of my system to, more or less, the 84\% of Android devices currently active.
	\item \textit{Stability}: system must be always available, and able to offer all its services. For example I should avoid possible system crashes during the delivery of a message from a device to another. In addition, data must be durable and not lost for any reasons.
	\item \textit{Availability}: the services must be always accessible in time. In case of failures it is possible for the user to manually restart it to be again usable.
	\item \textit{Reliability}: since data are shared among devices, reliability is essential. Users can base their actions on other users’ actions and on the status of the devices. Moreover, I assume that the memory where data are stored is stable.
	\item \textit{Efficiency}: within software development framework, efficiency means
	using as few resources as possible. Thus, the system will provide data
	structures and algorithms aimed to maximize efficiency. I will also try to
	use well known patterns reusing as many pieces of code as possible, taking
	care of avoiding any anti-patterns.
	\item \textit{Extensibility}: my application must provide a design where future updates
	are possible. It will be developed in such a way that the addition of new
	functionalities will not require radical changes to the internal structure and
	data flow.
	\item \textit{Maintainability}: also modifications to a code that already exists have to be
	taken into account. For this reason the code must be easily readable and
	fully commented.
	\item \textit{Security}: Using a networking service security is always required. The fact that
	the system will be available only on LANs is the first step in this direction.
\end{itemize}
\subsection{Used Technologies}
This other subsection is an overview of the tools I have used to develop the
Android application.\\
Android applications are written using Java code and any API level of the Android framework. The \textit{Liquid Android APK} is been developed by using standard Android development tools and libraries without the need to rely on any third party API library. In particular I decided to make these choices.
\begin{itemize}
	\item \textit{Android API level 19}, as already explained several times I want that my system can be installed on the largest number of devices, so it is a compromise between the great innovation introduced starting from this API level, and the number of devices which can execute API level 19 apps.
	\item \textit{Android NSD library}, it is a standard Android library I am using to register, discover and resolve my network service. It is an implementation of Zeroconf and it is compatible with other implementation such as Apple's bonjour.
	\item \textit{Standard Java libraries}, I decided to use only libraries included in standard Java development kit distribution, the most important are :
	\begin{itemize}
		\item \textit{org.json}, used to manage the JSON file, which are the messages exchanged by the devices using my application;
		\item \textit{java.net}, containing the classes which are useful to implement the socket network communication.
	\end{itemize}	
\end{itemize}
Technically, I used an IDE to help in development, in particular AndroidStrudio based on IntelliJ IDEA, a
modern solution released by Jet Brains. It contains modern tools to check Java
compile time errors flagging them, and tools to check run time errors with a
complete and verbose stack trace.Finally,
it supports various VCS (Versioning control systems), to push the code inside
repositories and to have a complete overview of commits and forks. I chose Git,
one of the most famous VCS, and GitHub as repository to save my code.
\subsection{Implementation}
I have developed a fully working Android application called \textit{Liquid Android}. I have build the application using the API library i have created and already presented in \ref{API}, so the structure of my code follows exactly the one presented with the \textit{Liquid Android API} UML model, in \figurename~\ref{fig:4.7}. The components and the methods i have used, to create the application, are exactly the one presented in that figure with little modifications and adaptations.
\subsubsection{Code organization}
\begin{figure}[h]
	\centering
	\includegraphics[width=.45\textwidth]{package}
	\caption{Code organization}
	\label{fig:5.1}
\end{figure}
As anticipated, the code is organized following the \textit{MVC} design pattern, so the \textit{controller components} are all contained in the \textit{Logic} package, while the \textit{UI components} are left inside the \textit{Main} package of the application- Other components typical of the Android development framework are left in their standard locations, such as the XML file containing the \textit{application manifest}.

\subsubsection{Implicit Intents to listen}
\lstinputlisting[language=XML , caption=Liquid Android MainActivity Manifest example, label=code:5.1]{Codici/manifest.xml}
In \lstlistingname~\ref{code:5.1} there is part of the manifes, of my application, showing some common intent filter the \textit{Liquid Android} app can listen to. In figure we can see that is the MainActivity of the application which declare itself capable of managing intents to take a picture, send and email or open a map. By adding any intent filter to the manifest of the application Liquid Android can listen and forward, automatically any kind of Android implicit intent. This snippet of code is the way in which the \text{FR1} is practically implemented.\\\\
In the following section I want to describe my system in action, providing application's screenshots, UML diagrams, working tests and use cases.
\section{Working Demo}
This section is intended to show the reader the \textit{Liquid Android Application} while it is working. After the structure and the implementation of the application were explained it is necessary to show the finished work. As anticipated, my application
is simply a proof of concept of how it is possible to use a group of Android devices, as they were executing a single distributed operating system using well known  Android mechanisms. Uers should not worry about substrates, they can control everything with a single and simple standard Android UI.\\
\subsection{Live Test Cases}
%
% ------------------------------------------------------------------------ %
% !TEX encoding = UTF-8 Unicode
% !TEX TS-program = pdflatex
% !TEX root = ../Tesi.tex
% !TeX spellcheck = en_US
% ------------------------------------------------------------------------ %
%
% ------------------------------------------------------------------------ %
% 	CONCLUSIONS AND FUTURE WORKS
% ------------------------------------------------------------------------ %
%
\chapter{Conclusions and Future Works}
%
\label{cap:conclusions}
%
% ------------------------------------------------------------------------ %
%
This chapter aims to analyze what has been done in order to give feedback
and understand if the initial objectives have been reached.\\
The chapter is divided into two parts, the first is the conclusion of the work, with
the analysis of what I have reached with my idea and my implementation; the
second part is focused on future implementation.
\section{Conclusions}
Liquid Android middleware aims to enrich the Android OS functionalities by transforming it in to a fully working distributed operating system. My work aims to differentiate itself from existing commercial solutions, seen at the beginning of the solution chapter \ref{cap:proposedsolution}, offering a both low cost and easy to deploy solution using a distributed approaches where mobile devices are actors and computing nodes.\\
Liquid Android middleware offers:
\begin{itemize}
	\item automatic network construction and maintenance, when devices are connected in LAN,
	\item same intent resolution mechanism of the standard Android OS,
	\item intent conversion language and automatic tools to perform the conversions,
	\item same standard Android security mechanisms,
	\item development API, easy to use and to be extended to create other purposes distributed systems.
\end{itemize}
Theoretically speaking the goal was reached: having created the networking structure and having defined a new language suitable for the purpose to distribute intents over the network. In this way the Android OS become a distributed OS over a LAN, because any task which can be done by creating implicit intent can be encapsulated in a message and sent, to be executed, on any device with the service active in the network.The research part
was the study and the definition of mechanisms able to let communicate different devices, using a standard mobile operating system. I have defined the steps needed to perform this kind of extension with any existing mobile operating system. My steps, in \ref{cap:proposedsolution} can be used indeed to create similar solutions for other kind of devices. The creation of the middleware and of reliable connections was more part of the implementation, where I did not introduce innovations in the used technologies. That is because reliable and powerful technologies assure the right behavior of the whole architecture.\\
Tests performed in this work aimed to stress the whole system simulating a common scenario with a few number of devices involved: they shown good overall performance of the framework and its well behavior in terms of scalability and dynamicity. Real Android devices behaved as expected under test conditions and the test app built on purpose was always fully usable.\\
I think the work which has been done is positive, interesting and promising: the
idea of having distributed mobile operating system to let users ,use their devices in an environment as they were a single big device, by sending task to any of them connected with my service, is certainly a good feature nowadays in which this kind of devices are becoming more numerous and less expensive. Moreover it is possible to exploit, such a system, to take advantage of the increasing computing capability of mobile devices, and let them cooperate respecting standard OS mechanisms.
\section{Future Works}
In this section I would like to briefly report and talk about the future developments which can be done on my work, in order to extend and make the framework as complete possible.
My thesis aims to find a unequivocal and uniform way to extend mobile operating system to give them distributed system functionalities. In particular I want to let any kind of device in a network cooperate with each other to perform task. Obviously It was not possible for
reason of time, and complexity to develop a system to be put on the market nowadays, compatible with all the devices. Thus I selected Android devices because they are the most common, cheap, and they use an open source OS. My work can be part of a more complex and complete architecture which can be widely adopted. Following the the very same steps I have done, it would be possible to define a standard language definition to let any mobile devices communicate and cooperate under the same LAN. My JSON solution can be extended and easily used to adapt also IOS and Windows Phone similar intent mechanism, to reach the global goal.




%
% -----------------------------END------------------------------------- %
%
% ------------------------------------------------------------------------ %
% 	BACKMATTER
% ------------------------------------------------------------------------ %
%
\cleardoublepage
%
\backmatter
%
% ------------------------------------------------------------------------ %
% !TEX encoding = UTF-8 Unicode
% !TEX TS-program = pdflatex
% !TEX root = ../Tesi.tex
% !TEX spellcheck = it-IT
% ------------------------------------------------------------------------ %
%
% ------------------------------------------------------------------------ %
% 	FIGURES COPYRIGHT
% ------------------------------------------------------------------------ %
%
\chapter{Figures Copyright}
%
\label{cap:figurescopyright}
%
% ------------------------------------------------------------------------ %
%
\section*{Chapter 2: State of Art}



\section*{Chapter 4: Solution}


%

The images that are not explicitly listed are made by me and no copyright is needed.
 
% -----------------------------END------------------------------------- %
%
% ------------------------------------------------------------------------ %
% !TEX encoding = UTF-8 Unicode
% !TEX TS-program = pdflatex
% !TEX root = ../Tesi.tex
% !TEX spellcheck = it-IT
% ------------------------------------------------------------------------ %
%
% ------------------------------------------------------------------------ %
% 	BIBLIOGRAFIA
% ------------------------------------------------------------------------ %

\cleardoublepage
\nocite{*}	% anche riferimenti non citati
\printbibliography

 % ------------------------------------------------------------------------ %
% ------------------------------------------------------------------------ %
% 	END DOCUMENT
% ------------------------------------------------------------------------ %
%
\end{document}
%
% ------------------------------------------------------------------------ %